\section{Projective modules}

\begin{editor}
  This appendix has been solely written by the author as a fact collection,
  it is not an official part of the lecture.
\end{editor}

\begin{definition}
  Let $R$ be a ring.
  An $R$-module $P$ is projective if one of the following equivalent
  conditions holds:
   \begin{enumerate}[p]
    \item The functor $\Hom_R(P, \dash)$ is exact.
    \item For every surjection $\mu \colon  M \twoheadrightarrow N$
      and every morphism $p\colon P \to N$ lifts to a morphism
      $\hat{p} \colon  P \to  M$:
      \[
      \begin{tikzcd}[column sep = tiny]
      & P \ar[swap,dashed]{dl}{\hat{p}} \ar{dr}{p} \\
      M \ar[swap]{rr}{\mu} & & N
      \end{tikzcd}
      \]
    \item Every short exact sequence
      \[
      0 \to  L \xrightarrow{λ} M \xrightarrow{\mu} \xrightarrow{P} \to 0
      \] 
      splits.
    \item $P$ is a direct summand of a free module,
      that is, there exists a module $K$ and a free module  $F$,
      such that  $F \cong P \oplus K$.
  \end{enumerate}
\end{definition}

\begin{lemma}
  \label{lm:facts-projective-modules}
  Let $P_1$ and $P_2$ be projective modules.
  Then, the following hold:
  \begin{enumerate}[h]
    \item
      $P_1 \oplus P_2$ and $P_1 \otimes _R P_2$
      are projective modules.
    \item Every direct summand of $P_1$ is projective.
    \item For every $R$-module  $M$, there exists a projective resolution
       \[
      \ldots \to P_3 \to P_2 \to P_1 \to P_0 \to  M \to  0
      \] 
  \end{enumerate}
\end{lemma}

\begin{lemma}
  Let $P$ be finitely generated $R$-module.
  Then  $P$ is projective if and only if  $P$ is a direct summand
  of  $R^n$ for some $n\in \mathbb{N}$.
\end{lemma}

\begin{lemma}
  Let $P$ be an  $R$-module.
  Then, the following hold:
   \begin{enumerate}[h]
     \item Let $P$ be free. Then  $P$ is projective.
     \item Let $P$ be projective. Then  $P$ is flat. 
     \item Let $P$ be flat. If  $R$ is noetherian and  $P$ is
       finitely generated, then  $P$ is projective.
  \end{enumerate}
\end{lemma}

\begin{definition}
  A chain complex $C_\chainbullet $ is called \vocab{projective}
  if all $C_n$ are projective modules.
\end{definition}
