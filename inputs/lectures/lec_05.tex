%! TEX root = ../../master.tex
\lecture[Properties of compact (connected) manifolds, in particular orientability criterion and existence of the fundemental class.]{Mi 20 Apr 2022}{Existence of the fundamental class}

\begin{refproof}{lm:top-homology-of-manifold-relative-to-cocompact-subset}
  \begin{claim}
    If the statement is true for $K_1$, $K_2\subset M$ and their
    intersection $K_1\cap K_2$,
    then the statement also holds for $K_1\cup K_2$.
  \end{claim}
  \begin{proof}
    Since $M$ is Hausdorff, the complements  $M \setminus K_i$
    are open and thus $(M, M\setminus K_1, M\setminus K_2)$
    is excisive and by
    \autoref{thm:mayer-vietoris-union-intersection-of-subspaces}
    we get the Mayer-Vietoris-sequence
    \begin{IEEEeqnarray}{rCl}
      \label{eq:mayer-vietoris-cocompact-subsets}
      \ldots
      &
      \to
      &
      H_{n+1}(M, M \setminus (K_1 \cap K_2); N)
      \to
      H_n(M,M\setminus (K_1 \cup K_2);N)
      \\
      &
      \to
      &
      H_n(M, M\setminus K_1; N) \oplus H_n(M, M\setminus K_2; N)
      \to 
      H_n(M, M \setminus (K_1 \cap K_2);N)
      \to \ldots
    \end{IEEEeqnarray}
    $1)$ follows immediately.
    For $2)$ note that $H_{d+1}(M, M\setminus (K_1\cap K_2;N)$
    vanishes and thus the map into the sum is injective.
    Let $u\in H_d(M, M\setminus (K_1 \cup K_2); N)$ be
    mapped to $(u_1,u_2)$ by this.
    By commutativity, an application of $2)$ for $K_1$
    and $K_2$ separately shows that $u_1$ and $u_2$ are trivial,
    hence by injectivity also $u$ is trivial.
  \end{proof}
  Inductively, we immediately get the following:
  \begin{claim}
    \label{cl:lemma-holds-for-union-of-compact-subsets}
    If the statement is true for compact subsets
    $K_1, \dotsc, K_n$ and all their intersections
    $\bigcap_{j\in J\subset \set{ 1,\dots, n } } K_j$,
    then the statement holds for their union $K_1 \cup \dotsb \cup K_n$.
  \end{claim}
  Next, we show the following:
  \begin{claim}
    The statement holds for $M=\mathbb{R}^d$ and any compact $K\subset M$.
  \end{claim}
  \begin{proof}
    Omitted for now.
    \todo{add proof for this}
  \end{proof}
  Now, let $K$ be general.
  By compactness of $K$, we can find compact subsets $K_i$ such that
  $K = \bigcup_{i =1}^n K_i$, where $K_i$ is contained in a subset
  $U_i$ homeomorphic to $\mathbb{R}^d$.
  By excision, we get that
  \[
    H_n(M, M\setminus K_i;N) \cong H_n(U_i, U_i\setminus K;N)
  \]
  and we deduce the statement for each $K_i$ separately.
  Thus, by
  \autoref{cl:lemma-holds-for-union-of-compact-subsets},
  the lemma follows.
\end{refproof}

\begin{lemma}
  \label{lm:inclusion-is-injective-on-homology-for-compact-neighborhood-in-manifold}
  Let $M$ be a $d$-dimensional manifold and $N$ be an $R$-module.
  If $K\subset M$ is compact and connected, the map
  \[
  (i_x)_*\colon H_d(M, M\setminus K;N) \to H_d(M, M\setminus \set{ x } ;N)
  \]
  is injective for every $x\in K$.
\end{lemma}

\begin{proof}
%  Pick $u\in H_d(M, M\setminus K; N)$ that lies in the kernel of $(i_x)_*$.
  Omitted for now.
  \todo{proof this}
\end{proof}

\begin{theorem}
  \label{thm:unique-lifts-on-compact-subsets-in-manifold-and-iso-for-connected}
  Let $M$ be a $d$-dimensional manifold with an $R$-orientation
  $\set{ μ_x } $, $K\subset M$ a compact subset and $R$ be a ring.
  Then, the following hold:
  \begin{enumerate}[h]
    \item There is a unique $μ_K\in H_d(M, M\setminus K;R)$
      mapping to $μ_x$ for each $x\in K$.
    \item If $K$ is connected, the map
      \[
        H_d(M, M\setminus K; R) \to H_d(M, M\setminus \set{ x } ;R)
      \]
      is an isomorphism for each $x\in K$.
  \end{enumerate}
\end{theorem}

\begin{definition}*
 Let $M$ be a $d$-dimensional manifold with an $R$-ori\-en\-ta\-tion
 $\set{ μ_x }$.
 We say that a subset $U\subset M$ has the \vocab{lifting property}
 if there exists some $μ_U$ such that for each $x\in X$,
 the map
 \[
   (i_x)_* \colon H_d(M, M\setminus U;N) \to H_d(M, M\setminus \set{ x } ;N)
 \]
 sends $μ_U$ to $μ_x$.
 Note that we do not require that such a $μ_U$ is unique.

 In particular, the definition of a manifold says that
 each point has a neighborhood with the lifting property.
\end{definition}

\begin{remark}*
  Note that the if $U$ satisfies the lifting property,
  then also each subset of $U$ does.
\end{remark}

\begin{proof}
  First, note that if a $\mu_K$ as in $1)$ exists,
  then it is already unique by
  \autoref{lm:top-homology-of-manifold-relative-to-cocompact-subset},
  as the difference of two such $μ_K$, $μ_K'$ maps to zero for each
  $x\in K$.

  For each $x\in K$, pick a lift $μ_{U_x}$ of $\set{ μ_x }$
  for some neighborhood $U_x$.

  As these cover $K$, by compactness and intersecting with $K$ we may pick
  finitely many $K_1, \dotsc, K_n$ that cover $K$,
  where each $K_i$ is contained in some $U_x$ and in particular has
  the lifting property.
  Let $L$ and $L'$ be two such compact subsets,
  then again considering the Mayer-Vietoris-sequence
  \eqref{eq:mayer-vietoris-cocompact-subsets},
  we see that $(μ_L, μ_{L'})$ maps to zero under
  \[
    H_d(M, M\setminus L; R) \oplus H_d(M, M\setminus L'; R)
    \to 
    H_d(M, M\setminus (L \cup L'); R)
  \] 
  by applying the uniqueness condition we already proved.
  By exactness, we can thus find some $μ_{L\cup L'}$
  that is a lift on $L\cup L'$.

  By induction, we deduce that also $K = K_1 \cup \dotsb \cup K_n$
  has the lifting property.
  This proves $1)$.

  For $2)$, note that as $μ_K$ maps to $μ_x$ and $μ_x$ generates
  $H_d(M, M\setminus \set{ x } ; R)$, the map is surjective.
  But as $K$ is connected, by
  \autoref{lm:inclusion-is-injective-on-homology-for-compact-neighborhood-in-manifold}
  we deduce that the map is also injective, hence an isomorphism.
\end{proof}

\begin{remark}
  As we will soon introduce manifolds with boundary,
  we will refer to \enquote{closed manifolds} instead of
  \enquote{compact manifolds} from now on.
  This terminology is also more common in literature.
\end{remark}

\begin{theorem}
  \label{thm:fundamental-class-exists-and-top-homology-is-r-iff-orientable}
  Let $M$ be a $d$-dimensional closed manifold and $R$ be a ring.
  Then, the following hold:
  \begin{enumerate}[h]
    \item For each $R$-orientation $\set{ μ_x } $ of $M$,
      there is a unique element $[M]\in H_d(M; R)$ such that
      $[M]$ maps to $μ_x$ for every $x\in M$.
    \item
      If $M$ is connected, the following are equivalent:
      \begin{enumerate}[e]
        \item $M$ is orientable
        \item $H_d(M ; \mathbb{Z}) \cong \mathbb{Z}$
        \item $H_d(M; \mathbb{Z})$ is nontrivial.
      \end{enumerate}
  \end{enumerate}
\end{theorem}

\begin{proof}
  Part $1)$ directly follows from
  \autoref{thm:unique-lifts-on-compact-subsets-in-manifold-and-iso-for-connected}
  by setting $K=M$.
  For $2)$ note that $(1)\implies (2)$ by statement $2)$ of
  \autoref{thm:unique-lifts-on-compact-subsets-in-manifold-and-iso-for-connected}.
  As we also clearly have $(2) \implies(3)$,
  it suffices to show $(3) \implies (1)$.

  So assume that $H_d(M)$ is non-trivial.
  As the map
  \[
    H_d(M) \to H_d(M, M\setminus \set{ x } ) \cong \mathbb{Z}
  \]
  is injective for each $x\in M$ by
  \autoref{lm:inclusion-is-injective-on-homology-for-compact-neighborhood-in-manifold},
  we deduce that $H_d(M)\cong \mathbb{Z}$
  (but not that the above map is necessarily an isomorphism).
  Pick a generator $μ\in H_d(M)$.
  Now, for $x\in M$ define $μ_x$ as the unique generator in
  $H_d(M, M\setminus \set{ x } )$ such that the image of $μ$ is a positive
  multiple of $μ_x$.

  We claim that $\set{ μ_x } $ defines an orientation.
  For arbitrary $x$, pick a neighborhood $U$ such that
  $H_d(M, M\setminus U) \to  H_d(M, M\setminus \set{ x } )$
  is an isomorphism.
  \todoquestion{why does this exist?}
  By picking $μ_U$ such that the image of $μ$ is a positive multiple
  of $μ$, we see that $μ_U$ does indeed map to $μ_y$ for each $y\in U$.
\end{proof}

\begin{definition}[Fundamental class]
  \label{def:fundamental-class}
  Let $R$ be a ring and $M$ be a closed,
  $d$-dimensional manifold with an $R$-orientation.
  The unique element $[M]$ from part $1)$ of
  \autoref{thm:fundamental-class-exists-and-top-homology-is-r-iff-orientable}
  is called the \vocab{fundamental class} of $M$. 
\end{definition}
