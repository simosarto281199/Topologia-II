%! TEX root = ../../master.tex
\lecture[Non-naturality of Künneth splitting. Eilenberg-Zilber. Homological cross product. Excisive triads. The Künneth theorem for relative homology. Cohomological cross product.]{Mi 06 Apr 2022}{Eilenberg Zilber}

\begin{example}[The Künneth splitting is not natural]
  \label{ex:künneth-splitting-is-not-natural}
  Let $p\colon \mathbb{R}\mathbb{P}^2 \to S^2$ be the map that collapses the $1$-skeleton.
  From the cellular chain complex, we see that then the map
  $\id\times p\colon \mathbb{R}\mathbb{P}^2\times \mathbb{R}\mathbb{P}^2
  \to \mathbb{R}\mathbb{P}^2\times S^2$
  induces an isomorphism on $H_3(\dash ; \mathbb{Z})$.
  A straightforward computation shows that the induced map
  of Künneth sequences (ignoring the outer zeros) has the form
  \[
    \begin{tikzcd}[column sep = small]
      0
      \ar{d}
      \ar{r}
      &
      H_3(\mathbb{R}\mathbb{P}^2 \times \mathbb{R}\mathbb{P}^2; \mathbb{Z})
      \ar{d}{\cong}
      \ar{r}{\cong}
      &
      \Tor_1^{\mathbb{Z}}(H_1(\mathbb{R}\mathbb{P}^2; \mathbb{Z}), H_1(\mathbb{R}\mathbb{P}^2;\mathbb{Z}))
      \ar{d}
      \\
      H_1(\mathbb{R}\mathbb{P}^2 ;\mathbb{Z}) \tensor H_2(S^2;\mathbb{Z})
      \ar{r}{\cong}
      &
      H_3(\mathbb{R}\mathbb{P}^2\times S^2;\mathbb{Z})
      \ar{r}{\cong}
      &
      0
    \end{tikzcd}
  .\]
  This shows that the splitting can't be natural, since it would have
  to factorize over $0$.
\end{example}

We would like to generalize the
\autoref{cor:künneth-for-cw-complexes}
to arbitrary spaces.
As this means returning to singular homology instead of cellular homology again,
however, the chain complex of the product space $X\times Y$ is no longer
isomorphic to the product of chain complexes of $X$ and  $Y$
(compare
\autoref{rm:cellular-chain-complex-of-product-of-cw-complexes-is-tensor-product}
).

However, it is still a homotopy equivalence, which will allow us to deduce
Künneth nonetheless.


\begin{theorem}[Eilenberg-Zilber]
  \label{thm:eilenberg-zilber}
  Let $R$ be a ring and let  $X$, $Y$ be spaces.
  There is a natural chain homotopy equivalence
  \[
    \mu\colon
    C_\chainbullet ^{\sing}(X;R) \chaintensor _R C_\chainbullet ^{\sing}(Y;R)
    \xrightarrow{\simeq} 
    C_\chainbullet ^{\sing}(X\times Y;R)
  .\] 
\end{theorem}

\begin{theorem}[Künneth theorem]
  \label{thm:künneth-theorem}
  Let $R$ be a  principal ideal domain
  and let $X$,  $Y$ be topological spaces.
  Then, for all $n\geq 0$, there is a natural short exact sequence
    \begin{IEEEeqnarray*}{rCClCl}
      0
      &
      \to
      &
      \directsum_{p+q=n}
      &
      H_p(X;R) \tensor H_q(Y;R)
      &
      \to
      &
      H_n(X\times Y;R)
      \\
      &
      \to
      &
      \directsum_{p+q=n-1}
      &
      \Tor_1^R(H_p(X;R), H_q(Y;R))
      &
      \to
      &
      0
      .
    \end{IEEEeqnarray*}
    Moreover, the sequence splits, but the splitting is not natural.
\end{theorem}
\begin{proof}*
  Use the \nameref{thm:künneth-for-chain-complexes} and apply
  it to to the singular chain complexes of $X$ and  $Y$.
  Then use
  \nameref{thm:eilenberg-zilber}
  to obtain the singular homology group $H_n(X\times Y;R)$.
\end{proof}

Next, we want to examine the relative case of homology.
For this,
we introduce some notation and need the following lemma:

\begin{notation}
  \label{not:csing-abbreviation}
  For brevity, we will often write $\Csing(\dash)$
  for the singular chain complex $C_\chainbullet ^{\sing}(\dash;R)$.
\end{notation}

\begin{notation}
  Let $(X,A)$ and  $(Y,B)$ be pairs of spaces.
  We denote by
  $\Csing\set{ A\times Y, X\times B } $ 
  the image of
  $\Csing(A\times Y) \oplus \Csing(X\times B)$
  in
  $\Csing(X\times Y)$.
\end{notation}

\begin{lemma}
  \label{lm:eilenberg-zilber-relative-spaces}
  Let $(X,A)$ and  $(Y,B)$ be pairs of spaces
  and let $R$ be a  principal ideal domain.
  We have the following commutative diagrams with short exact rows:
  \[
    \begin{tikzcd}[column sep = tiny]
      0
      \ar{r}
      &
      \Csing(A) \chaintensor_R \Csing(Y)
      \ar{r}
      \ar{d}[swap]{\simeq}{\mu}
      &
      \Csing(X)\chaintensor _R \Csing(Y)
      \ar{r}
      \ar{d}[swap]{\simeq}{\mu}
      &
      \faktor{\Csing(X)}{\Csing(A) \chaintensor _R \Csing(Y)}
      \ar{r}
      \ar{d}[swap]{\simeq}
      &
      0
      \\
      0
      \ar{r}
      &
      \Csing(A\times Y)
      \ar{r}
      &
      \Csing(X\times Y)
      \ar{r}
      &
      \faktor{\Csing(X\times Y)}{\Csing(A\times Y)}
      \ar{r}
      &
      0
    \end{tikzcd}
  \]
  and
  \[
    \begin{tikzcd}[column sep = tiny]
      0
      \ar{r}
      &
      \faktor{\Csing(X)}{\Csing(A)} \chaintensor \Csing(B)
      \ar{r}
      \ar{d}[swap]{\simeq}
      &
      \faktor{\Csing(X)}{\Csing(A)}\chaintensor \Csing(Y)
      \ar{r}
      \ar{d}[swap]{\simeq}
      &
      \faktor{\Csing(X)}{\Csing(A)} \chaintensor \faktor{\Csing(Y)}{\Csing(B)}
      \ar{r}
      \ar{d}[swap]{\simeq}
      &
      0
      \\
      0
      \ar{r}
      &
      \faktor{\Csing(X\times B)}{\Csing(A\times B)}
      \ar{r}
      &
      \faktor{\Csing(X\times Y)}{\Csing(A\times Y)}
      \ar{r}
      &
      \faktor{\Csing(X\times Y)}{\Csing \set{ A\times Y, X\times B }}
      \ar{r}
      &
      0.
    \end{tikzcd}
  \]
\end{lemma}


\begin{lemma}*
  \label{lm:quasi-isomorphism-of-free-chain-complexes-is-homotopy-equivalence}
  Let $C_\chainbullet $ and $D_\chainbullet $ be projective chain complexes
  and let $\mu\colon C_\chainbullet \to D_\chainbullet $
  be a quasi-isomorphism.

  Then, $\mu$ is already a homotopy equivalence.
\end{lemma}
\begin{proof}*
  This was proved in the last lecture of last term.
\end{proof}

\begin{refproof}{lm:eilenberg-zilber-relative-spaces}
  For the first diagram,
  the top row is exact since $\Csing(Y)$ is free.
  The bottom row is exact by construction.
  From the naturality of $\mu$,
  we get that the left square commutes,
  and thus the right-hand map is induced as the map on cokernels.
  
  Using the induced natural long exact sequence on homology,
  by the Five Lemma, we obtain the right-most map is an isomorphism
  on homology.
  Then, by
  \autoref{lm:quasi-isomorphism-of-free-chain-complexes-is-homotopy-equivalence} 
  we get that it is also a homotopy equivalence.

  For the bottom diagram,
  the top row is exact since  $\Csing(X)/\Csing(A)$ is free,
  the bottom row is exact by construction.
  The first two maps are the induced ones in the first diagram.
  By naturality of $\mu$ and naturality of the induced maps of the cokernel,
  we get that the left square commutes.

  As before, by using the long exact sequence in homology,
  the Five Lemma and
  \autoref{lm:quasi-isomorphism-of-free-chain-complexes-is-homotopy-equivalence},
  we deduce that the induced map on cokernels is a homotopy equivalence.
\end{refproof}


\begin{definition}+[Homological cross product]
  \label{def:homological-cross-product}
  Define the \vocab{homological cross product}
  \[
    \times \colon
    H_p(X,A;R) \tensor H_q(Y,B;R)
    \to
    H_n(X\times Y, X\times B \cup A\times Y; R)
  \]
  as the composition
  \begin{IEEEeqnarray*}{rCl}
    H_p(\Csing (X,A)) \tensor H_q(\Csing(Y,B))
    &
    \to
    &
    H_{p+q}(\Csing(X,A) \chaintensor \Csing(Y,B))
    \\
    &
    \xrightarrow{\mu, \cong} 
    &
    H_{p+q}( \Csing(X\times Y) / \Csing\set{ A\times Y, X\times B }
    \\
    &
    \to
    &
    H_{p+q} ( \Csing(X\times Y) / \Csing (A\times Y \cup X\times B))
    .
  \end{IEEEeqnarray*}
\end{definition}

Note that the second map $\mu$ is an isomorphism by
\autoref{lm:eilenberg-zilber-relative-spaces}
and the second is just the natural map.
We want to understand when this natural maps in in fact an isomorphism as well,
for this recall

\begin{definition}[Excisive triad]
  \label{def:excisive-triad}
  A triad $(X,A,B)$ (i.e.~$A$ and  $B$ are subspaces of  $X$)
  is  \vocab{excisive} if
  \[
    H_n(A,A\cap B;R) \to H_n(A\cup B, B ;R)
  \]
  is an isomorphism for all $n$.
\end{definition}

\begin{example}+
  A triad $(X,A,B)$ is excisive if
   \begin{itemize}
    \item $A$,  $B$ are open in  $X$
    \item  $X$ is a  CW-complex and $A$,  $B$ are subcomplexes.
  \end{itemize}
\end{example}

\begin{lemma}
  \label{lm:excisive-triad-quasi-isomorphism-relative-chain-complex}
  Let $(X,A,B)$ be an excisive triad.
  Then, the natural map
   \[
     \faktor{\Csing(X)}{\Csing \set{ A,B } }
     \to
     \faktor{\Csing(X)}{\Csing(A\cup B)}
  \]
  is a quasi-isomorphism.
\end{lemma}

\begin{corollary}+[Relative Eilenberg Zilber]
  \label{cor:relative-eilenberg-zilber}
  Let $(X,A)$ and  $(Y,B)$ be pairs of spaces such that
  $(X\times Y, A\times Y, X\times B)$ is an excisive 
  triad.
  Then, there is a natural chain homotopy equivalence
  \[
    \mu \colon
    \Csing(X,A) \chaintensor \Csing(Y,B)
    \xrightarrow{\simeq}
    \Csing(X\times Y, A\times Y \cup X\times B)
  .\] 
\end{corollary}
\begin{proof}
  Follows as a direct implication of
  \autoref{lm:eilenberg-zilber-relative-spaces}
  and
  \autoref{lm:excisive-triad-quasi-isomorphism-relative-chain-complex} 
\end{proof}

\begin{editor}
  We did not explicitly state
  \autoref{cor:relative-eilenberg-zilber}
  in the lecture, but given its subsequent uses,
  I think it is fine to state it separately.

  Otherwise, just always use
  \autoref{lm:eilenberg-zilber-relative-spaces} 
  and
  \autoref{lm:excisive-triad-quasi-isomorphism-relative-chain-complex}.
\end{editor}

\begin{refproof}{lm:excisive-triad-quasi-isomorphism-relative-chain-complex}
  Consider the short exact sequence
  \[
    0
    \to
    \Csing(A\cup B) / \Csing \set{ A,B }
    \to 
    \Csing(X) / \Csing \set{ A,B } 
    \xrightarrow{α} 
    \Csing(X) / \Csing( A \cup B)
    \to 
    0
  .\]
  The induced long exact sequence on homology
  \[
    \begin{tikzcd}[column sep = tiny]
      &
      \ldots
      \connectingmark[Z]
      \ar{r}
      &
      H_{n+1} (X, A\cup B)
      \connectingmap[Z]{\partial }
      \\
      H_n( \Csing(A\cup B) / \Csing \set{ A,B } )
      \ar{r}
      &
      H_n( \Csing(X) / \Csing \set{ A,B } )
      \connectingmark[Y]
      \ar{r}
      &
      H_n(X, A\cup B)
      \connectingmap[Y]{\partial }
      \\
      H_{n-1}( \Csing(A\cup B) / \Csing \set{ A,B } )
      \ar{r}
      &
      \ldots
    \end{tikzcd}
  \]
  shows that $α$ is a quasi-isomorphism if (and only if)
  $H_*(\Csing(A\cup B) / \Csing \set{ A,B } ) = 0$.
  But now considering the short exact sequence
  \[
    0
    \to
    \Csing(A) / \Csing(A\cap B)
    \to
    \Csing(A\cup B) / \Csing(B)
    \to
    \Csing(A\cup B) / \Csing \set{ A,B } 
    \to 
    0
  ,\]
  the corresponding long exact sequence on homology
  \[
    \begin{tikzcd}[column sep = tiny]
    &
    \ldots
    \connectingmark[Y]
    \ar{r}
    &
    H_{n+1} ( \Csing(A\cup B) / \Csing \set{ A,B }  )
    \connectingmap[Y]{\partial}
    \\
    H_n ( A, A\cap B )
    \ar{r}
    &
    H_n ( A\cup B, B )
    \connectingmark[Z]
    \ar{r}
    &
    H_n ( \Csing(A\cup B) / \Csing \set{ A,B }  )
    \connectingmap[Z]{\partial}
    \\
    H_{n-1} ( A, A\cap B )
    \ar{r}
    &
    \ldots
    \end{tikzcd}
  \]
  shows that this is the case if (and only if)
  the triad $(X,A,B)$ is excisive.
\end{refproof}

\begin{corollary}[Künneth theorem]
  Let $R$ be a  principal ideal domain
  and let $(X,A)$  and $(Y,B)$ be pairs of spaces such that
  $(X\times Y, A\times Y, X\times B)$ is an excisive triad.
   
  Then, for all $n\geq 0$,
  there is a natural short exact sequence
  \begin{IEEEeqnarray*}{rCl}
    0
    &
    \to
    &
    \directsum_{p+q=n}
    H_p(X,A;R) \tensor H_q(Y,B;R)
    \\
    &
    \xrightarrow{\times } 
    &
    H_n(X\times Y, X\times B \cup A\times Y;R)
    \\
    &
    \to
    &
    \directsum_{p+q=n-1}
    \Tor_1^R(H_p(X,A;R), H_q(Y,B;R))
    \\
    &
    \to
    &
    0
    .
  \end{IEEEeqnarray*}
  Moreover, the sequence splits, but the splitting is not natural.
\end{corollary}

\begin{proof}
  We will use the
  \nameref{thm:künneth-for-chain-complexes}
  for the chain complexes
  $\Csing(X,A)$ and $\Csing(Y,B)$ again,
  giving us a short exact sequence.
  \begin{IEEEeqnarray}{rCClCl}
    \label{eq:künneth-relative-ses-raw}
    0
    &
    \to
    &
    \directsum_{p+q=n}
    &
    H_p(X,A;R) \tensor H_q(Y,B;R)
    &
    \to 
    &
    H_n( \Csing(X,A) \chaintensor \Csing(Y,B))
    \IEEEnonumber
    \\
    &
    \to
    &
    \directsum_{p+q=n-1}
    &
    \Tor_1^R(H_p(X,A;R), H_q(Y,B;R))
    &
    \to
    &
    0
    .
  \end{IEEEeqnarray}
  Now using \nameref{cor:relative-eilenberg-zilber},
  we get the following induced isomorphism on homology
  \[
    H_n(\Csing(X,A;R) \chaintensor \Csing(Y,B;R))
    \xrightarrow{\cong}  
    H_n(\Csing(X\times Y, X \times B \cup A\times Y;R))
  .\]
  Using this together with
  \eqref{eq:künneth-relative-ses-raw}
  yields the desired short exact sequence.
  Note that the
  \nameref{def:homological-cross-product}
  appears as the first map in the resulting short exact sequence
  by definition.
\end{proof}

\begin{notation}
  Similar to
  \autoref{not:csing-abbreviation},
  we will write
  $\Csing*(\dash)$
  for the singular cohomology
   $C^{\chainbullet}_{\sing}(\dash; R)$.

   More generally,
   for a given chain complex $D_\chainbullet $,
   let us denote by $D^{\chainbullet}$ the corresponding
   cochain complex $\Hom(D_\chainbullet , M)$.
\end{notation}

\begin{definition}+
  Let $C^{\chainbullet}$ and $D^{\chainbullet}$ be $R$-cochain complexes.
  Similar to
  \autoref{def:tensor-product-chain-complexes},
  define the \vocab{tensor product of cochain complexes}
  as the cochain complex with modules
  \[
    (C^{\chainbullet} \chaintensor D^{\chainbullet})_n
    \coloneqq 
    \bigoplus_{p+q=n} C^{p} \tensor D^q
  \]
  and the differential
  \[
    d(a \chaintensor b) \coloneqq
    da \tensor  b + (-1)^{\chaindimension{a}} a \tensor db.
  \] 
\end{definition}
\begin{well-definedness}
  Same as in
  \autoref{def:tensor-product-chain-complexes}.
\end{well-definedness}

\begin{definition}+[Cohomological cross product]
  Let $(X,A)$ and  $(Y,B)$ be pairs of spaces such that
   $(X\times Y, A\times Y, X\times B)$ is an excisive triad.
  We define the \vocab{cohomological cross product} as the composition
  of
  \begin{IEEEeqnarray*}{Cl}
    &
    H_p(\Csing*(X,A)) \tensor H_q(\Csing*(Y,B))
    \\
    \to
    &
    H_{p+q}(\Csing*(X,A) \tensor \Csing*(Y,B))
    \\
    \to
    &
    H_{p+q}((\Csing(X,A) \tensor \Csing(Y,B))^{\chainbullet})
    \\
    \xleftarrow{\mu^*, \cong}
    &
    H_{p+q}( \left( \Csing(X\times Y) / \Csing \set{ A\times Y, X\times B }  \right) ^{\chainbullet} )
    \\
    \xleftarrow{\cong}
    &
    H_{p+q}((\Csing(X\times Y) / \Csing( A\times Y \cup X\times B))^{\chainbullet})
    \\
    =
    &
    H^{p+q}(X\times Y, A\times Y \cup X\times B; R)
    .
  \end{IEEEeqnarray*}
\end{definition}
\begin{well-definedness}
  Note that the third and fourth map exist as the induced map on
  cohomology of
  \nameref{cor:relative-eilenberg-zilber}.
\end{well-definedness}

\begin{definition}+[Cup product]
  \label{def:cup-product}
  Let $(X,A)$ and  $(Y,B)$ be pairs of spaces such that
   $(X\times Y, A\times Y, X\times B)$ is excisive.
  Define the \vocab{cup product} on singular cohomology
  as the composition
  \begin{IEEEeqnarray*}{rCl}
    \cup \colon H^p(X,A;R) \tensor H^q(X,B;R)
    &
    \xrightarrow{\times }
    &
    H^{p+q}(X\times X, X\times B \cup A\times X ;R)
    \\
    &
    \xrightarrow{\Delta^*}
    &
    H^{p+q}(X, A\cup B;R)
    ,
  \end{IEEEeqnarray*}
  where $\Delta\colon  X \to X\times X$ is the diagonal map as usual.
\end{definition}
