%! TEX root = ../../master.tex
\lecture[]{Mi 04 Mai 2022}{Untitled}

\begin{theorem}
  If $L(p,q) \simeq L(p,q')$,
  then there exists  $n$ such that  $q\equiv \pm n^2q' \mod p$.
\end{theorem}

\begin{proof}
  Any such homotopy equivalence
  $h\colon  L(p,q) \to L(p,q')$ has degree $\pm 1$.
  Pick a generator $b\in H^1(L(p,q') ; \mathbb{Z}/p)$,
  then also $a\coloneqq  h^*(b)$ is a generator of $H^1(L,pq); \mathbb{Z}/p)$,
  and we may compute that
  \begin{IEEEeqnarray*}{rCl}
    t_q
    &\simeq&
    \left< a \cup β(a), [L(p,q)] \right> 
    =
    \left< h^*b \cup β(h^*b), [L(p,q)] \right> 
    \\
    &=&
    \left< h^* (b\cup β(b), L[(p,q)] \right> 
    =
    \left< b\cup β(b), h_* [L(p,q)] \right> 
    \\
    &=&
    \pm \left< b\cup β(b), [L(p,q')] \right> 
    \simeq
    t_{q'}
  \end{IEEEeqnarray*}
  Hence by
  \autoref{lm:relation-lense-spaces-t-q-is-q-times-t-1}
  we deduce that $qt_1 \simeq t_q \simeq t_{q'} \simeq q't_1$.
  But now as $t_1$ is a generator of $\mathbb{Z}/p$,
  by the definition of $\simeq$ we deduce that
  $q \equiv  \pm n^2q' \mod p$
  for some $n\in \mathbb{N}$.
\end{proof}


\begin{corollary}
  There exist orientable, closed, connected, $3$-dimensional manifolds
  $M$ and  $N$, such that
  \begin{enumerate}[h]
    \item $M$ and  $N$ have the same fundamental group
    \item  $M$ and  $N$ have the same homology groups
    \item  $M$ and  $N$ are not homotopy equivalent.
  \end{enumerate}
\end{corollary}

\begin{proof}
  We can check that $2$ is not  $\pm n^2 \mod 5$ for any $n$,
  thus  $1 \not \simeq 2 \in \mathbb{Z}/5 / \simeq$.
  This immediately gives us that the lense spaces $L(5,1)$ and  $L(5,2)$
  are not homotopy equivalent by
  \autoref{thm:homotopy-equivalent-lense-spaces-same-invariant}.

  We have already computed that they have the same fundamental
  and homology groups.
\end{proof}

\begin{remark}
  \begin{enumerate}[1.]
    \item 
      The converse of
      \autoref{thm:homotopy-equivalent-lense-spaces-same-invariant}
      is also true.
      That is, if $q \simeq q'$,
      then also  $L(p,q) \simeq L(p,q')$  are homotopy equivalent.
    \item
      It can be shown that two lense spaces $L(p,q)$ and  $L(p,q')$
      are homeomorphic if and only if  $q' \equiv \pm q^{\pm 1} \mod p$.
  \end{enumerate}
\end{remark}



\section{Additive and abelian categories}

We will study some category theory for now.
Although this is not strictly necessary for the rest of the lecture,
we will show that the category of $R$-modules is complete and cocomplete,
that is, it has all (small) limits and colimits.

While this can also be shown by hand directly,
this gives some more general approach.


\begin{definition}+
  An object in a category $\mathcat{C}$ is called a \vocab{zero object}
  if it is both initial and terminal.
\end{definition}

\begin{remark}+
  Note that \enquote{the} zero object is unique up to unique isomorphism,
  as terminal and initial objects are.
\end{remark}

\begin{notation}
  If $\mathcat{C}$ is a category with a zero object,
  we usually denote this object by $0$.
  Also note that for any two objects  $X_0$ and $X_1$,
  there is a unique map
  \[
  X_0 \to  0 \to  X_1
  \]
  as $0$ is both terminal and initial,
  which we want to call the  \vocab{zero map}
  and denote by $0$ as well.
\end{notation}


\begin{definition}
  A category $\mathcat{A}$ is called \vocab{semi-additive} if
  \begin{enumerate}[p]
    \item $\mathcat{A}$ has a zero object
    \item $\mathcat{A}$ has all finite products and coproducts
    \item For any two objects $X_0$ and $X_1$, the canonical map
      \[
      X_0 \coprod X_1 \to  X_0 \prod X_1
      ,\]
      induced by the maps  $X_i \xrightarrow{\id} X_i$
      and $X_i \xrightarrow{0} X_{i+1}$,
      is an isomorphism.
  \end{enumerate}
\end{definition}

\begin{remark}
  For semi-additive categories, we usually denote the (co)product of
  $X$ and  $Y$ by  $X \oplus Y$ and call it the \vocab{biproduct}
  or the \vocab{sum} of $X$ and  $Y$. 

  As we will see in the exercises, we can use this to equip $\Hom(X,y)$
  with a canonical structure of a commutative monoid.
\end{remark}

\begin{remark}*
  We will briefly talk about \vocab{enriched} categories now.
  As a precise definition of an enriched category is quite long
  (although not too complicated),
  we won't state it here in full length and rather refer to the
  nLab by
  \cite{nlab:enriched+category}.

  For us, we want to think of a category $\mathcat{C}$ enriched
  over some category $\mathcat{V}$ as some category whose
  $\Hom$-objects  $\mathcat{C}(X,Y)$ are objects in $\mathcat{V}$,
  and where composition of maps
  $\mathcat{C}(X,Y) \times \mathcat{C}(Y,Z) \to \mathcat{C}(X,Z)$
  is given in some \enquote{nice} way.
  
  To be more precise, $\mathcat{V}$ has to have some notion of
  a tensor product, and we want composition to be a map
  $\mathcat{C}(X,Y) \otimes \mathcat{C}(Y,Z) \to \mathcat{C}(X,Z)$.
\end{remark}

\begin{definition}
  A category $\mathcat{A}$ is called \vocab{pre-additive} if it is enriched
  over the category of abelian groups.

  To be precise, this means that every $\Hom$-set $\mathcat{A}(A,B)$ 
  is an abelian group and composition
  $\mathcat{A}(X,Y) \oplus \mathcat{A}(Y,Z) \to \mathcat{A}(X,Z)$
  is bilinear.
\end{definition}

\begin{remark}*
  The notion of \enquote{tensor products} in  $\Ab$
  is just the direct sum (or biproduct) $\oplus$.
\end{remark}

\begin{lemma}
  Let $\mathcat{A}$ be a semi-additive category.
  If the monoids $\Hom(X,Y)$ are abelian groups for all  $X$,  $Y$,
  $A$ is additive.

  A functor between additive categorise is additive if and only
  if it preserves all biproducts.
\end{lemma}

\begin{proof}
  Exercise.
\end{proof}

\begin{example}
  The category of $R$-modules is an additive category.
  Moreover, the full subcategories of finitely generated,
  projective or free modules are additive as well.
\end{example}

\begin{definition}[Kernel]
  \label{def:kernel}
  Let $\mathcat{A}$ be an additive category.
  The \vocab{kernel} of a map $f\colon X\to Y$
  is the equalizer of $f$ and  $0$.
  That is, the kernel is a pair  $(\ker f, \varphi)$
  where $\varphi \colon \ker f \to X$, such that the following
  universal property holds:
  \[
    \begin{tikzcd}
      &
      A
      \ar[dashed, swap]{dl}{\exists !}
      \ar{d}{g}
      \ar{dr}{0}
      \\
      \ker f
      \ar{r}{\varphi}
      &
      X
      \ar{r}{f}
      &
      Y
    \end{tikzcd}
  \]
\end{definition}

\begin{definition}+[Cokernel]
  \label{def:cokernel}
  Define the \vocab{cokernel} of a morphism $f\colon X\to Y$
  in an additive category $\mathcat{A}$ to be the coequalizer
  of $f$ and  $0$.
\end{definition}

\begin{definition}
  A \vocab{pre-abelian} category is an additive category
  sch that every morphism has a kernel and cokernel.
\end{definition}

\begin{example}
  The category of $R$-modules is pre-abelian.

  The full subcategory of finitely generated $R$-modules
  is pre-abelian if  $R$ is noetherian
  (in particular for  principal ideal domains $R$).

  The category of free or projective $R$- modules is usually
  not pre-abelian,
  as e.g.~the cokernel of $\mathbb{Z} \xrightarrow{\cdot 2} \mathbb{Z}$
  is not projective.
\end{example}

\begin{proposition}
  \label{prop:limits-in-pre-abelian-category}
  Let $\mathcat{A}$ be a pre-abelian category.
  Then, $\mathcat{A}$ has all finite limits and colimits.

  If $\mathcat{A}$ has all products (coproducts),
  then $\mathcat{A}$ has all limits (colimits).
\end{proposition}

\begin{proof}
  We check that in an additive category,
  the equalizer (coequalizer) of $f$,  $g\colon X \to Y$
  is the same as the equalizer (coequalizer) of $f-g$ and $0$.

  Hence, by definition, pre-abelian categories have all equalizers
  and coequalizers.

  Now, the proposition immediately follows from
  \autoref{lm:products-and-equalizers-give-limits},
  as pre-abelian categories have finite limits and colimits by definition.
\end{proof}

\begin{lemma}
  \label{lm:products-and-equalizers-give-limits}
  Let $\mathcat{C}$ be a category that has all (finite) products (coproducts)
  and equalizers (coequalizers).
  Then, $\mathcat{C}$ has all (finite) limits (colimits).
\end{lemma}

\begin{proof}
  We only prove the statement for products,
  dualizing yields the case for coproducts.

  Assume that $\mathcal{F}\colon \mathcat{I} \to  \mathcat{C}$ is a diagram.
  Now consider the diagram
  \[
  \begin{tikzcd}
    &
    &[3em]
    &
    \mathcal{F}A
    \ar{d}{f}
    \\
    \eq(s,t)
    \ar{r}{\varphi }
    &
    \prod\limits_{A \in \Ob(\mathcat{I})} \mathcal{F}A
    \ar[shift left, near start]{r}{s}
    \ar[shift right, swap, near start]{r}{t}
    \ar[bend left = 30]{urr}{\pr_A}
    \ar[bend right = 30, swap]{drr}{\pr_B}
    &
    \smashoperator{\prod\limits_{f\colon A \to B \in  \Mor (\mathcat{I}) }} \; \mathcal{F}B
    \ar{r}{\pr_f}
    &
    \mathcal{F}B
    \\
    &
    &
    &
    \mathcal{F}B
    \ar[swap]{u}{=}
  \end{tikzcd}
  ,\]
  where we mean that the top diagram commutes with $s$
  and the bottom with $t$.
  Note that this defines $s$ and  $t$ by the universal properties
  of the product uniquely.
  Also note that all occurring products are finite iff  $\mathcat{I}$ is finite.

  Now, some standard universal property checking yields that
  the equalizer of $s$ and  $t$ has the universal property of
  $\lim \mathcal{F}$.
\end{proof}

\begin{corollary}
  The category of $R$-modules is complete and cocomplete.
\end{corollary}

\begin{example}
  Working through the proof, we see that the colimit
  of some diagram $\mathcal{M}\colon \mathcat{I} \to \Mod_R$
  is given by the quotient
  \[
  \faktor{\directsum_{i \in I} M_i}{\simeq}
  \]
  where the equivalence relation is generated by
  $\mathcal{M}(f)(x) \sim y$ for $f\colon M_i \to M_j$
  and $x\in M_i$, $y\in M_j$.
\end{example}
