%! TEX root = ../../master.tex
\lecture[]{Mo 02 Mai 2022}{Untitled}

\todoquestion{why equality in definition of the form}

\begin{oral}
  We did not see this yet, but the intersection pairing
  is closely related to intersections on manifolds.
  One can show that, counting signs, the intersection pairing
  counts the number of intersections of two
  cohomology classes.
\end{oral}

\begin{definition}
  \label{def:unimodular-form-on-manifold-of-even-dimension}
  Let $R$ be a principal ideal domain,
  $M$ an $R$-oriented, connected, closed,
  $d$-dimensional manifold with $d=2k$.
  Then the \vocab{intersection form} is the unimodular form
  $\overline{s_{k,M,R}}$.
\end{definition}

\begin{remark}
  The intersection form is symmetric if $k$ is even
  and antisymmetric if $k$ is odd.

  This follows from the graded commutativity of the cup product.
\end{remark}

\todo{forms}

\begin{definition}[Signature of forms]
  \label{def:signature-of-form}
  Let $s$ be a symmetric form over $\mathbb{R}$
  on a vector space $V$.
  The \vocab{signature} of $s$ is the triple
  $(p_{+}, p_{-}, p_0) \in \mathbb{N}^3$,
  with $p_{+}$, $p_{-}$ the sum of the dimensions of the
  positive and negative eigenspaces and $p_0$ the dimension
  of the eigenspace with eigenvalue $0$.

  If $s$ is unimodular, then $p_0=0$.
  In this case, we define the \vocab{signature}  as
  $σ(s) = p_{+} - p_{-} \in \mathbb{Z}$.
\end{definition}

\begin{example}*
  Consider $s_{1, S^2 \times S^2, \mathbb{R}}$,
  represented by the matrix
  $\begin{pmatrix} 1 & 0 \\ 0 & 1 \end{pmatrix} $.

  This has $(x,x)$ with Eigenvalue $1$
  and $(x,-x)$ with Eigenvalue $-1$.
  Thus,
  \[
    σ(S^2\times S^2) = 1 - 1 = 0
  \] 
\end{example}

\todo{hyperbolic form}
\todo{matrix gives intersections of the generators!}

\begin{definition}[Signature of manifold]
  \label{def:signature-of-manifold}
  Let $M$ be a closed, oriented, connected,
  $4k$-dimensional manifold $M$.
  The \vocab{signature} of $M$ is defined to be the
  signature of $s_{2k, M, \mathbb{R}}$.

  If $M$ is not connected, then define
  \[
    σ(M) \coloneqq \sum_{\text{$M_i$ conn. comp.}} σ(M_i)
  .\]
  If the dimension of $M$ is not divisible by $4$,
  define $σ(M^d) \coloneqq 0$.
\end{definition}

\todo{verify definition non-connectedness}

\begin{oral}
  At some point, we will introduce manifolds with boundary.
  One can then define that two manifolds of dimension $k$
  are \vocab{bordant} if their disjoint union is the boundary
  of some other manifold.

  It turns out that the signature of a manifold is a bordism
  invariant.
\end{oral}

\begin{oral}
  Also note that forms are not invariant under bordism:
  $S^2\times S^2$ is bordant to $S^4$ via the $5$-dimensional
  manifold $S^2 \times D^3$.

  As the form on $S^4$ is trivial, the form do not agree.
\end{oral}




\subsection{$3$-dimensional Lens spaces}

\begin{oral}
  One can actually study Lens spaces in all odd dimension.
  However, in the lecture, we will restrict ourselves to
  dimension $3$.
\end{oral}

\begin{definition}
  Let $S^3\subset \mathbb{C}^2$ be the unit sphere and
  $p$ a prime number.
  Denote by $ω = e^{\frac{2 \pi i}{p}}$ a primitive $p$-th root of unity.
  For $0 < q < p$ (actually $q$ coprime to $p$),
  define
    \begin{equation*}
    T_q: \left| \begin{array}{c c l} 
    S^3 & \longrightarrow & S^3 \\
    (u,v) & \longmapsto &  (ωu, ω^qv)
    \end{array} \right.
    .
  \end{equation*}
  Note that this gives a free $\mathbb{Z}/p$ action on $S^3$.
  The \vocab{Lense space}, denoted $L(p,q)$, is the quotient space
  of $S^3$ by this action.
\end{definition}

\begin{lemma}+
  $L(p,q)$ is an orientable, $3$-dimensional manifold
  with $\pi_1(L(p,q)) \cong \mathbb{Z}/p$.
\end{lemma}

For
\[
  0
  \to
  \mathbb{Z}
  \xrightarrow{\cdot p}
  \mathbb{Z}
  \to 
  \faktor{\mathbb{Z}}{p}
  \to
  0
\]
we get a long exact sequence in cohomology, given by
\[
  \begin{tikzcd}[column sep = tiny]
  &
  \ldots
  \connectingmark[Y]
  \ar{r}
  &
  H^{n-1} ( L; \mathbb{Z}/p )
  \connectingmap[Y]{\partial}
  \\
  H^n ( L; \mathbb{Z} )
  \ar{r}
  &
  H^n ( L; \mathbb{Z} )
  \connectingmark[Z]
  \ar{r}
  &
  H^n ( L; \mathbb{Z}/p )
  \connectingmap[Z]{\partial}
  \\
  H^{n+1} ( L; \mathbb{Z} )
  \ar{r}
  &
  \ldots
  \end{tikzcd}
.\]
\todo{Bockstein morphism $β$}

Now, by the UCT, we see that $H^1(L; \mathbb{Z})$ is trivial,
and thus $H^2(L; \mathbb{Z})  \cong H^2(L; \mathbb{Z}/p) \cong \mathbb{Z}/p$.
Again, by the UCT, we see that $H^1(L; \mathbb{Z}/p) \cong \mathbb{Z}/p$.

Thus, the Bockstein isomorphism $β^1$ and the reduction
morphism in second cohomology are isomorphism.

\begin{definition}+
  We define 
  \[
    β
    \coloneqq
    \red \circ β^1
    \colon 
    H^1(L;\mathbb{Z}/p)
    \xrightarrow{\cong}
    H^2(L,\mathbb{Z}/p)
  \] 
\end{definition}

Let $a\in H^1(L; \mathbb{Z}/p)$ be a generator.
Then also $β(a) \in H^2(L;\mathbb{Z}/p)$ is a generator
and thus
\[
  \left< a, β(a) \cap [L] \right> \in \mathbb{Z}/p
\]
is a generator, as $s_{1, L(p,q), \mathbb{Z}/p}$ is a perfect pairing.

As for now, this generator depends both on the choice of $a$
and the choice of an orientation on $L$.

Orientation of $L$ gives a $\pm 1$.
Let $b$ be another generator of $H^1(L; \mathbb{Z}/p)$,
then $b=na$ for some $n$ coprime to $p$.

Then,
\[
  \left< b, β(b) \cap [L] \right>
  =
  \left< na, β(na) \cap [L] \right>
  =
  n^2 \left< a, β(a) \cap [L] \right> 
.\]

Hence, $\left< a, β(a) \cap [L] \right> $ is well-defined up to
multiplication with $\pm n^2$ with $n$ coprime to $p$.

\todoquestion{pairing: $\mathbb{Z}$ orientation, view $\mathbb{Z}/p$ as $\mathbb{Z}$-module}

Let $\simeq$ be the equivalence relation on $\mathbb{Z}/p$ given by
$i \simeq j$ if $\exists n$ such that $i = \pm n^2 j$.

Let $t_q \coloneqq \left< a, β(a) \cap [L(p,q)] \right> \in \mathbb{Z}/p / \simeq$.

\begin{lemma}
  We have $t_q \simeq q t_1$.
\end{lemma}

\begin{proof}
  For $v = r\cdot e^{i\theta} \in \mathbb{C}$ define
  $v^{(k)} \coloneqq re^{ik\theta}$.
  Define
    \begin{equation*}
    \varphi : \left| \begin{array}{c c l} 
    S^3 & \longrightarrow & S^3 \\
    (u,v) & \longmapsto &  (u, v^{(q)})
    \end{array} \right.
    .
  \end{equation*}
  We claim that $\varphi $ is equivariant, for this compute
  \[
    \varphi T_1(u,v) = \varphi(\omega u, \omega v)
    =
    (\omega u, (\omega v)^{(q)})
    =
    T_q(u, v^{(q)})
    =
    T_q \varphi (u,v)
  \]

  Thus, $\varphi$ induces a map $\psi \colon  L(p,1) \to L(p,q)$
  on the corresponding Lense spaces.

  \begin{claim}
    $\psi $ is an isomorphism on $\pi_1$.
  \end{claim}
  \begin{proof}
    Let $α$ be a non-trivial element in $\pi_1(L(p,1))$.
    Then, there is a path $z$ from some $x_0\in S^3$ to
    $T_1^k(x_0)$ such that its image in $L(p,1)$
    represents $α$.
    $k$ is coprime to $p$.
    Under $\varphi $ this maps to a path $\varphi \circ z$
    from $\varphi (x_0)$ to $\varphi (T_1^k(x_0)) = T_q^k(\varphi (x_0))$.
    This represents $\psi _*(α)$.
  \end{proof}
  Thus, $\psi _*$ is an isomorphism on $H_1$,
  hence also
  \[
    \psi ^* \colon  H^1(L(p,q), \mathbb{Z}/p)
    \xrightarrow{\cong}
    H^1(L(p,1), \mathbb{Z}/p)
  \]
  is an isomorphism (UCT).
  The map $\varphi  $ has degree $q$,
  e.g.~by counting preimages, or it is the double suspension
  of the map $v \mapsto v^q$ on $S^1$.

  Now, by commutativity of
  \[
    \begin{tikzcd}
      H_3(S^3; \mathbb{Z})
      \ar[swap]{d}{\cdot p}
      \ar{r}{\cdot q}
      &
      H_3(S^3 ; \mathbb{Z})
      \ar{d}{\cdot p}
      \\
      H_3(L(p,1);\mathbb{Z})
      \ar[swap]{r}{\cdot q}
      &
      H_3(L(p,q);\mathbb{Z})
    \end{tikzcd}
    ,
  \]
  (check that the covering has degree $p$)
  we deduce that also $\psi $ has degree $q$.

  Now, let $a\in H^1(L(p,q);\mathbb{Z}/p)$.
  Then
  \[
    q\cdot t_q \simeq \left< a, β(a) \cap q[L(p,q)] \right>
    =
    \left< a, β(a) \cap \psi _* [L(p,1)] \right> 
    =
    \left< \psi ^* (a \cup β(a)), [L(p,1)] \right> 
    =
    \left< \psi ^*(a) \cup β(\psi ^*(a)), [L(p,1)] \right> 
    =
    \left< \psi ^*(a), β(\psi ^*(a)), [L(p,1)] \right> 
  .\] 
  But now, $t_q \simeq q^2t_q \simeq qt_1$
  and we're done.
\end{proof}

\todo{covering condition, sheet 11 GeoTopo}
