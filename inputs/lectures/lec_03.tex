%! TEX root = ../../master.tex
\lecture[]{Mo 11 Apr 2022}{Eilenberg Zilber}

\begin{theorem}[Künneth in cohomology]
  Let $R$ be a PID.
  Let  $(X,A)$ and  $(Y,B)$ be spaces
  such that $(X \times Y, A\times Y, X\times B)$ is excisive
  and $(X,A)$ is of finite type,
  i.e.~ $H_n(X,A;R)$ is finitely generated for all $n$.
  Then, there is a natural short exact sequence
    \begin{IEEEeqnarray*}{rCCl}
      0
      &
      \to
      &
      \directsum_{p+q=n}
      &
      H^p(X,A;R) \tensor H^q(Y,B;R)
      \xrightarrow{\times } 
      H_n(X\times Y, X\times B \cup A\times Y;R)
      \\
      &
      \to
      &
      \directsum_{p+q=n+1}
      &
      \Tor_1^R(H^p(X,A;R), H^q(Y,B;R))
      \to
      0
      .
    \end{IEEEeqnarray*}
    Moreover, the sequence splits, but the splitting is not natural.
\end{theorem}

\begin{oral}
  The notion of \enquote{finite type} encountered here is strictly
  weaker than the notion of a finite type $CW$-complex.
\end{oral}

\begin{proof}
  Finite type is needed because in general,
  \[
  C^{\chainbullet}(X,A) \otimes _R C^{\chainbullet} (Y,B)
  \to
  \left( C_\chainbullet (X,A) \otimes _R C_\chainbullet (Y,B) \right) ^{\chainbullet}
  \]
  is \emph{not} a chain homotopy equivalence.

  If $C_\chainbullet (X,A)$ is finitely generated (free) in each degree,
  then the above map \emph{is} an isomorphism.
  Using
  $\Hom_R(R^n, R) \otimes _R A \xrightarrow{\cong}  \Hom_R(R^n,A)$
  and
  $\Hom(A, \Hom(B,R)) \cong \Hom(A \otimes B, R)$.

  If $(X,A)$ is of finite type,
   $C_\chainbullet (X,A)$ is chain homotopy equivalent
   to a chain complex
   $C_\chainbullet $ such that $C_n$ is finitely generated
   for all  $n$:
    \[
   \begin{tikzcd}
     C^*(X,A) \otimes C^(Y,B) \ar[phantom]{d}[rotate=90]{\simeq} \ar{r}{} & (C_*(X,A) \otimes _R C_*(Y,B))^* \ar[phantom]{d}[rotate=90]{\simeq} \\
     C^* \otimes C^*(Y,B) \ar[swap]{r}{} & (C_* \otimes _R C_*(Y,B))^*
   \end{tikzcd}
   \]
\end{proof}

\section{Eilenberg-Zilber theorem}

Denote with $CH_+$ the category of chain complexes,
where  $C_n = 0$ for  $n<0$ with chain maps.

Let $F_*$, $G_*\colon \mathcat{C} \to  CH_+$ be functors
and $\varphi _*$, $\psi _*\colon  F_* \implies G_*$ natural transformations.

\begin{definition}
  A natural chain homotopy $s_* \colon  \varphi _* \simeq \psi _*$
  is a natural transformation $F_* \to  G_{* + 1}$
  such that
  \[
  d_{n+1}^G \circ  s_n + s_{n-1} \circ  d_n^F = \varphi _n - \psi _n
  \] 
\end{definition}

\begin{definition}
  A functor $F_n\colon \mathcat{C} \to \Mod_{\mathbb{Z}}$
  is called \vocab{free} if there exists a family
  $\left\{(B_{n,j}, b_{n,j} \mid  j\in J(n) \right\} $ 
  of objects $B_{n,j}$ (called \vocab{models}) in $\mathcat{C}$
  and elements $b_{n,j} \in F_n(B_{n,j})$,
  such that for all $X\in \Ob(\mathcat{C})$,
  \[
  \left\{F_n(f) (b_{n,j}) \mid  j\in J(n), f\in \Hom_{\mathcat{C}}(B_{n,j},X) \right\} 
  \]
  is a basis of $F_n(X)$.
\end{definition}

\begin{remark}
  Assume that $F_n\colon \mathcat{C} \to  \Ab$ is free,
  $G_n$ another functor  $\mathcat{C} \to  \Ab$.
  Then any natural transformation $\varphi _n \colon F_n \implies G_n$
  is determined by $\varphi _n(b_{n,j})$
  and any set of such values determines a natural transformation.
\end{remark}

\begin{proof}
  We note that
  \[
  \begin{tikzcd}
    F_n(B_{n,j)} \ar[swap]{d}{} \ar{r}{} & F_n(X) \ar{d}{} \\
   G_n(B_{n,j})  \ar[swap]{r}{} & G_n(X)
  \end{tikzcd}
  \]
  commutes, so any given transformation is determined by the $\varphi (b_{n,j})$.

  Conversely, we can define a transformation having such $b_{n,j}$
  in a canonical way.
  For checking commutativity, use the basis again.
\end{proof}

\begin{definition}
  We call $F_* \colon  \mathcat{C} \to  CH_+$ free if each $F_n$ is free.
  We call  $G_*$ acyclic with respect to the models
   $\left\{(B_{n,j}, b_{n,j})\right\} $ 
   if $H_m(G_*(B_{n,j})) = 0$ for all $m>0$ and all $n$,  $j$.
\end{definition}

\begin{example}
  Consider the functor
  $C_\chainbullet ^{\sing}(\blank ; \mathbb{Z})\colon \Top \to CH_+$.
  This is free with model $(\Delta^n, \id_{\Delta^n})$ in degree $n$.
  This is also acyclic.
\end{example}

\begin{example}
  Consider
  $C_\chainbullet ^{\sing}(\blank ; \mathbb{Z}) \otimes _{\mathbb{Z}} C_\chainbullet ^{\sing}
  (\blank ; \mathbb{Z}) \colon  \Top^2 \to  CH_+$.
  This functor is free with models given by
  \[
    \left\{(\Delta^k, \Delta^{n-k}), (\id_{\Delta^n} \otimes \id_{\Delta^{n-k}}) 
    \mid  0 \leq  k \leq  n\right\} 
  \]

  The functor $C_*^{\sing}(\blank \times  \blank ; \mathbb{Z}) \colon  \Top^2\to CH_+$
  is free with model
  \[
    \left\{(\Delta^n, \Delta^n), \Delta^n \xrightarrow{diag} \Delta^n \times \Delta^n \right\} 
  \]

  We have $H_m(\Delta^n \times \Delta^n) = 0 = H_m(\Delta^k \times \Delta^{n-k}$ 
  for all $m>0$.
  Thus,  $C_*^{\sing}(- \times - ; \mathbb{Z})$ is acyclic.

  Using Künneth, we see that also
  $C_*^{\sing}( \blank ; \mathbb{Z}) \otimes _Z C_*^{\sing} (\blank ; \mathbb{Z})$ 
  is acyclic (for both models).
\end{example}

\begin{theorem}
  Let $F_*$ be free and  $G_*$ be acyclic.
  For each natural transformation
  $\overline{\varphi } \colon  H_0 \circ  F_*
  \implies H_0 \circ  G_*$,
  there exists a natural transformation $\varphi \colon F_* \implies G_*$
  inducing $\overline{\varphi }$.

  Moreover, any two such transformations are naturally chain homotopic.
\end{theorem}

\todo{in $R$-Koeffizienten umformulieren}

\begin{proof}[Proof of Eilenberg-Zilber]
  It suffices to prove the theorem with $R = \mathbb{Z}$.
  As before, denote $C_*(X) \coloneqq C_*^{\sing}(X ; \mathbb{Z})$.

  We have
  \[
    H_0(C(X) \otimes C(Y))
    \stackrel{\text{Künneth}}{\cong}
    H_0(X) \otimes H_0(Y)
  \]
  We have a natural isomorphism
  $H_0(X \times Y) \cong H_0(X) \otimes H_0(Y)$,
  because both have pairs of path components as a basis.

  Hence we get natural transformations
  \[
    \psi \colon  C_*(\blank) \otimes C_*(\blank) \implies C_*( \blank \times  \blank )
  \]
  and
  \[
  \varphi  \colon  C_*( \blank \times  \blank ) \implies C_* ( \blank ) \otimes  C_* ( \blank )
  \]
  by 3.6

  Both compositions induce the identity on $H_0$.
  Thus, they are chain homotopic to the identity.

  Thus, $\psi $ is a chain homotopy equivalence.
\end{proof}

\begin{oral}
  note that we are using that both functors are not
  only acyclic for their own model but for both.
\end{oral}

\begin{question}
  In 3.6, do we mean $G_*$ acyclic wrt. to the models of  $F_*$.
  Yes.
\end{question}

\begin{proof}[of 3.6]
  We will specify $\varphi _n(b_{n,j})$ (and thus $\varphi _n$) inductively
  such that
  $d_n^G \varphi _n = \varphi _{n-1} d_n^F$.

  Choose $\varphi_0 ( b_{0,j})$ as a representative of
  $\overline{\varphi }([b_{0,j}])$.
  Assume that $\varphi _{n-1}$ is defined,
  we will now define $\varphi _n$.

  We have $\varphi _{n-1} ( d_n^F(b_{n,j})) \in  G_{n-1}(B_{n,j})$.
  \begin{claim}
    This is a boundary.
  \end{claim}
  \begin{proof}
    Consider first $n=1$.
    Then  $d_1^F(b_{1,j}) \in F_0 ( B_{1,j})$ is a boundary,
    hence trivial in homology.
    Thus, $\varphi _0 (d_1^F(b_{1,j}))$ is trivial in homology.
    Thus, it is a boundary.

    Consider $n\geq 2$, then
    \[
      d_{n-1}^G \varphi _{n-1} ( d_n^F ( b_{n,j})) = \varphi _{n-2} d_{n-1}^F d_n ^F ( b_{n,j}) = 0
    .\]
    As $G$ is acyclic, $\varphi _{n-1} d_n^F ( b_{n,j})$ is a boundary.
  \end{proof}
  Choose $g_{n,j} \in  G_n(B_{n,j})$ such that
  $d_n^G ( g_{n,j}) = \varphi _{n-1}(d_n^F(b_{n,j}))$
  and define $\varphi _n (b_{n,j}) = g_{n,j}$.
  It follows that $\varphi _*$ exists.

  Now assume that $\varphi _*$ and  $\psi _*$ both induce $\overline{\varphi }$.
  We will inductively define $S_n(b_{n,j}) \in  G_{n+1}(B_n,j)$
  such that $d_{n+1}^G \circ  s_n = s_{n-1} \circ  d_n^F = \varphi _n - \psi _n$.

  By definition, $\varphi _0(b_{0,j})$ and $\psi _0 (b_{0,j})$
  represent the same class in homology.
  Thus, $\varphi _0 ( b_{0,j}) - \psi _0 ( b_{0,j})$ is a boundary.
  Pick $c_{0,j} \in  G_1(B_{0,j})$ such that
  $d_1^G ( c_{0,j}) = \varphi _0(b_{0,j}) - \psi _0 (b_{0,j})$ 
  and define $s_0(b_{0,j}) = c_{0,j}$.

  Assume that $s_{n-1}$ is defined.
  \begin{IEEEeqnarray*}{rCl}
    d_n^G ( \varphi _n - \psi _n - s_{n-1} \circ  d_n^F)
    &=& \left( \varphi _{n-1} - \psi _{n-1} - \varphi _{n-1} + \psi _{n-1} + s_{n-1} nach d_{n-1}^F \right) d_nF = 0
  \end{IEEEeqnarray*}
  Since $G_*$ is acyclic, there exists a  $C_{n,j} \in  G_{n+1}(B_{n,j})$ 
  such that
  \[
    d_{n+1}^G ( G_{n,j}) = ( \varphi _n - \psi _n - s_{n-1} d_n^F) (b_{n,j})
  \]
  Define $s_n(b_{n,j)} \coloneqq  c_{n,j}$.
  This gives our desired chain homotopy
\end{proof}
