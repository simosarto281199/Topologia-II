%! TEX root = ../../master.tex
\lecture[]{Mo 11 Apr 2022}{Untitled}

test

\begin{theorem}
  Let $R$ be a PID.
  Let  $(X,A)$ and  $(Y,B)$ be spaces such that
   \begin{enumerate}[p]
    \item
      $(X \times Y, A\times Y, X\times B)$ is excisive
      and $(X,A)$ is of finite type, i.e.~ $H_n(X,a;R)$ is finitely generated for all $n$.
      Then there is a short exact sequence
      \[
      0 \to  \bigoplus_{p+q} H^p(X,A;R) \otimes H^q(Y,B;R)
      \xrightarrow{\times } H^n(X\times Y, X\times B \cup A\times Y;R)
      \to \bigoplus_{p+q=n-1} \Tor_1^R(H^p(X,A;R) H^q(Y,B,R))
      \]
      and the sequence splits, but not naturally.
  \end{enumerate}
\end{theorem}

\begin{oral}
  The notion of \enquote{finite type} encountered here is strictly
  weaker than the notion of a finite type $CW$-complex.
\end{oral}

\begin{proof}
  Finite type is needed because in general,
  \[
  C^{\chainbullet}(X,A) \otimes _R C^{\chainbullet} (Y,B)
  \to
  \left( C_\chainbullet (X,A) \otimes _R C_\chainbullet (Y,B) \right) ^{\chainbullet}
  \]
  is \emph{not} a chain homotopy equivalence.

  If $C_\chainbullet (X,A)$ is finitely generated (free) in each degree,
  then the above map \emph{is} an isomorphism.
  Using
  $\Hom_R(R^n, R) \otimes _R A \xrightarrow{\cong}  \Hom_R(R^n,A)$
  and
  $\Hom(A, \Hom(B,R)) \cong \Hom(A \otimes B, R)$.

  If $(X,A)$ is of finite type,
   $C_\chainbullet (X,A)$ is chain homotopy equivalent
   to a chain complex
   $C_\chainbullet $ such that $C_n$ is finitely generated
   for all  $n$:
    \[
   \begin{tikzcd}
     C^*(X,A) \otimes C^(Y,B) \ar[phantom]{d}[rotate=90]{\simeq} \ar{r}{} & (C_*(X,A) \otimes _R C_*(Y,B))^* \ar[phantom]{d}[rotate=90]{\simeq} \\
     C^* \otimes C^*(Y,B) \ar[swap]{r}{} & (C_* \otimes _R C_*(Y,B))^*
   \end{tikzcd}
   \]
\end{proof}

\section{Eilenberg-Zilber theorem}

Denote with $CH_+$ the category of chain complexes,
where  $C_n = 0$ for  $n<0$ with chain maps.

Let $F_*$, $G_*\colon \mathcat{C} \to  CH_+$ be functors
and $\varphi _*$, $\psi _*\colon  F_* \implies G_*$ natural transformations.

\begin{definition}
  A natural chain homotopy $s_* \colon  \varphi _* \simeq \psi _*$
  is a natural transformation $F_* \to  G_{* + 1}$
  such that
  \[
  d_{n+1}^G \circ  s_n + s_{n-1} \circ  d_n^F = \varphi _n - \psi _n
  \] 
\end{definition}

\begin{definition}
  A functor $F_n\colon \mathcat{C} \to \Mod_{\mathbb{Z}}$
  is called \vocab{free} if there exists a family
  $\left\{(B_{n,j}, b_{n,j} \mid  j\in J(n) \right\} $ 
  of objects $B_{n,j}$ (called \vocab{models}) in $\mathcat{C}$
  and elements $b_{n,j} \in F_n(B_{n,j})$,
  such that for all $X\in \Ob(\mathcat{C})$,
  \[
  \left\{F_n(f) (b_{n,j}) \mid  j\in J(n), f\in \Hom_{\mathcat{C}}(B_{n,j},X) \right\} 
  \]
  is a basis of $F_n(X)$.
\end{definition}

\begin{remark}
  Assume that $F_n\colon \mathcat{C} \to  \Ab$ is free,
  $G_n$ another functor  $\mathcat{C} \to  \Ab$.
  Then any natural transformation $\varphi _n \colon F_n \implies G_n$
  is determined by $\varphi _n(b_{n,j})$
  and any set of such values determines a natural transformation.
\end{remark}

\begin{proof}
  We note that
  \[
  \begin{tikzcd}
    F_n(B_{n,j)} \ar[swap]{d}{} \ar{r}{} & F_n(X) \ar{d}{} \\
   G_n(B_{n,j})  \ar[swap]{r}{} & G_n(X)
  \end{tikzcd}
  \]
  commutes, so any given transformation is determined by the $\varphi (b_{n,j})$.

  Conversely, we can define a transformation having such $b_{n,j}$
  in a canonical way.
  For checking commutativity, use the basis again.
\end{proof}

\begin{definition}
  We call $F_* \colon  \mathcat{C} \to  CH_+$ free if each $F_n$ is free.
  We call  $G_*$ acyclic with respect to the models
   $\left\{(B_{n,j}, b_{n,j})\right\} $ 
   if $H_m(G_*(B_{n,j})) = 0$ for all $m>0$ and all $n$,  $j$.
\end{definition}

\begin{example}
  Consider the functor
  $C_\chainbullet ^{\sing}(\blank ; \mathbb{Z})\colon \Top \to CH_+$.
  This is free with model $(\Delta^n, \id_{\Delta^n})$ in degree $n$.
  This is also acyclic.
\end{example}
