%! TEX root = ../../master.tex
\lecture[Reminder on projective modules, the $\Tor$ functor, universal coefficient theorem. The Künneth theorem for chain complexes and CW-complexes.]{Mo 04 Apr 2022}{The Künneth theorem}

\section{Reminder on UCT}

\begin{editor}
  We revised the following notions from last semester:
  \begin{itemize}
    \item Projective modules
    \item The $\Tor$ functor and its derived long exact sequences
    \item The UCT theorem for chain complexes and spaces
  \end{itemize}

  Here, we give a brief overview over these, refer to the notes of last
  semester for more details, especially for proofs.
\end{editor}

\begin{definition}[The $\Tor$ functor]
  Let $M$ and  $N$ be  $R$-modules.
  Pick a projective resolution $P$ of $M$ and define
  \[
  \Tor_n^R(M,N) \coloneqq  H_n ( P \otimes _R N )
  .\] 
\end{definition}

\begin{lemma}
  We have
  \begin{enumerate}[h]
    \item $\Tor_n^R(M,N) \cong \Tor_n^R (N,M)$.
    \item  If  $R$ is a principal ideal domain,
      then $\Tor_n^R = 0$ for $n\geq 2$.
    \item If $M$ is a projective module,
      then  $\Tor_n^R(M, \dash) = 0$ for $n\geq 1$.
  \end{enumerate}
\end{lemma}

\begin{proposition}[$\Tor$ exact sequence]
  \label{prop:tor-exact-sequence}

  Let $0 \to  A \to  B \to  C \to 0$  be a short exact sequence
  of $R$-modules.
  Then, there is a long exact sequence of $R$-modules,
  given by
  \[
    \begin{tikzcd}[column sep = tiny]
         \ldots \ar{r} & \Tor_2(B,M) \ar{r} \ar[phantom, d, ""{coordinate, name=Z1}]& \Tor_2(C,M)
        \ar[swap,dll,
        "δ",
        rounded corners,
        to path = {
            -- ([xshift=2em]\tikztostart.east)
            |- (Z1) [near end]\tikztonodes
            -| ([xshift=-2ex]\tikztotarget.west)
        -- (\tikztotarget)}] \\
         \Tor_1(A,M) \ar{r}
         &
         \Tor_1(B,M) \ar{r}
         \ar[phantom, d, ""{coordinate, name=Z2}]
         &
         \Tor_1(C,M)
        \ar[swap,dll,
        "δ",
        rounded corners,
        to path = {
            -- ([xshift=2em]\tikztostart.east)
            |- (Z2) [near end]\tikztonodes
            -| ([xshift=-2ex]\tikztotarget.west)
        -- (\tikztotarget)}] \\
        A \otimes  M \ar{r}
        &
        B \otimes  M \ar{r}
        &
        C \otimes M \ar{r}
        &
        0
    \end{tikzcd}
  \]
\end{proposition}

\begin{theorem}[UCT for chain complexes]
  \label{thm:uct-for-chain-complexes}

  Let $R$ be a  principal ideal domain,
  $C_\chainbullet $ a projective chain complex
  and $M$ an  $R$-module.

  Then, there is a natural short exact sequence
  \[
    \begin{tikzcd}[column sep = small, row sep = tiny]
      0 \ar{r}
      &
      H_n(C_\chainbullet ) \otimes M \ar{r}
      &
      H_n(C_\chainbullet  \otimes M) \ar{r}
      &
      \Tor_1(H_{n-1}(C_\chainbullet ), M) \ar{r}
      &
      0 \\
      &
      \left[ a \right] \otimes m \ar[mapsto]{r}
      &
      \left[ a \otimes m \right] 
    \end{tikzcd}
  \]
  Moreover, this sequence splits, but the splitting is not natural.
\end{theorem}

\begin{corollary}[UCT for spaces]
  \label{cor:uct-for-spaces}
  Let $R$ be a principal ideal domain, $X$ any space and  $M$ an  $R$-module.

  Then, there is a  short exact sequence
  \[
    0 \to  H_n(X;R) \otimes M \to  H_n(X ; M) \to \Tor_1(H_{n-1}(X;R),M) \to 0
  \] 
  Moreover, this sequence splits, but the splitting is not natural.
\end{corollary}

\begin{exercise}
  Repeat the story with the  $\Hom$ and  $\Ext$ functors for cohomology.
\end{exercise}




\section{The Künneth theorem}

In this section, we want to develop tools for computing the
homology of a product of spaces, i.e.~we want to know what
$H_\chainbullet  (X \times Y)$ is.

\begin{definition}
  \label{def:tensor-product-chain-complexes}
  Let $C_\chainbullet $ and  $D_\chainbullet $ be  $R$-chain complexes.
  We define the \vocab{tensor product of chain complexes}, denoted
  $C_\chainbullet \chaintensor  D_\chainbullet $,
  as the complex with modules
  \[
    (C_\chainbullet  \chaintensor D_\chainbullet )_n
    \coloneqq
    \directsum_{p + q = n} C_p \chaintensor D_q
  \]
  and differential
  \[
    d ( a \chaintensor b)
    \coloneqq
    da \chaintensor b + (-1)^{\chaindimension{a} } \cdot a \chaintensor db
  .\] 
\end{definition}

\begin{well-definedness}
  We have to check that the resulting complex
  does indeed satisfy $d^2 = 0$.
  The computation is straightforward:
  \begin{IEEEeqnarray*}{rCl}
    d^2 ( a \chaintensor b )
    &=&
    d ( da \chaintensor  b) + (-1)^{\chaindimension{a} } \cdot d ( a \chaintensor  db )
    \\
    &=&
    d^2a
    + (-1)^{\chaindimension{da}} da \chaintensor db
    + (-1)^{\chaindimension{a}} da \chaintensor  db
    + (-1)^{\chaindimension{a}} \cdot (-1)^{\chaindimension{a}} a \chaintensor d^2b
    \\
    &=&
    0
  \end{IEEEeqnarray*}
\end{well-definedness}

\begin{theorem}[Künneth for chain complexes]
  \label{thm:künneth-for-chain-complexes}
  Let $R$ be a principal ideal domain,
  $C_\chainbullet $ a projective chain complex and  $D_\chainbullet $ any chain complex.
  Then, there is a natural short exact sequence
  \[
     0
     \to
     \smashoperator{\directsum_{p + q = n}}
     H_p(C_\chainbullet ) \otimes H_q (D _\chainbullet )
     \to
     H_n (C_\chainbullet \otimes D_\chainbullet )
     \to 
     \smashoperator{\directsum_{p + q = n-1}}
     \Tor_1(H_p(C_\chainbullet ), H_q(D_\chainbullet ))
     \to
     0
  .\]
  Moreover, this sequence splits, but the splitting is not natural.
\end{theorem}

\begin{remark}[UCT as special case of \nameref{thm:künneth-for-chain-complexes}]
  Let $C_\chainbullet $ be a chain complex and $M$ an $R$-module. 
  Define the complex $D_\chainbullet $ as the complex
  \[
    \ldots \to  0 \to  M \to  0 \to  \ldots
  \]
  where $M$ is in degree $0$.
  Then the degreewise tensor with $M$ agrees
  with the tensor product of chain complexes with $D_\chainbullet $,
  i.e.~we have $C\chaintensor  D_\chainbullet  = C_\chainbullet \tensor M$.
  Furthermore, \autoref{thm:künneth-for-chain-complexes} specializes
  to the \nameref{thm:uct-for-chain-complexes}.
\end{remark}

\begin{refproof}{thm:künneth-for-chain-complexes}
  As usual, we denote the boundaries and cycles of $C_\chainbullet $ by
  \[
    Z_p \coloneqq \set{ a\in C_p \suchthat da = 0 },
    \qquad
    B_p \coloneqq \set{ da \suchthat  a \in C_{p+1} } 
  \]
  We have the short exact sequence
  \begin{equation}
    \label{eq:künneth-ses-cycles-boundaries}
    0 \to  Z_p \xrightarrow{ι}   C_p \xrightarrow{d}  B_{p-1} \to 0
  \end{equation}
  Since $R$ is a  principal ideal domain and $C_p$ is projective,
  $Z_p$ and  $B_p$ are also projective as direct summands
  of a projective module by
  \autoref{lm:facts-projective-modules}.

  Since $B_{p-1}$ is projective, we obtain the
  short exact sequence
  \[
    \Tor_1(B_{p-1}, D_q) = 0
    \to Z_p \tensor D_q
    \to C_p \tensor D_q
    \to B_{p-1} \tensor D_q
    \to 0
  \]
  by the \nameref{prop:tor-exact-sequence}.

  We regard $Z_\chainbullet $ and $B_\chainbullet $ as formal
  chain complexes, with $0$ as boundary maps in both cases.

  Then, with
  \eqref{eq:künneth-ses-cycles-boundaries},
  this gives a short exact sequence of chain complexes
  \begin{equation}
    \label{eq:künneth-ses-chain-complexes}
    0
    \to  Z_\chainbullet  \chaintensor D_\chainbullet 
    \to C_\chainbullet \chaintensor D_\chainbullet 
    \to  B_{\chainbullet-1} \chaintensor D_\chainbullet 
    \to  0
  \end{equation}
  Since $Z_\chainbullet $ and $B_{\chainbullet -1}$ are formal,
  we get
  \[
    Z_\chainbullet  \tensor D_\chainbullet
    \cong
    \directsum_p Z_p \tensor D_{\chainbullet -p},
    \qquad
    B_{\chainbullet -1} \tensor D_\chainbullet
    \cong
    \bigoplus_p B_{p-1} \chaintensor D_{\chainbullet -p}
  \]
  up to sign of the boundary maps $d$.
  Since $Z_p$ and  $B_{p-1}$  are projective and thus
  $\Tor_1(H_{n-1}(D_{\chainbullet -p}), Z_p)$
  as well as
  $\Tor_1(H_{n-1}(D_{\chainbullet -p}), B_{p-1})$
  vanish, by the
  \nameref{thm:uct-for-chain-complexes}
  we obtain that
  \begin{IEEEeqnarray*}{rCl}
    H_n (Z_\chainbullet  \chaintensor D_\chainbullet )
    &\cong&
    \directsum_{p+q = n} Z_p \tensor H_q(D_\chainbullet )
    \\
    H_n(B_{\chainbullet -1} \tensor D_\chainbullet )
    &\cong&
    \directsum_{p+q=n-1} B_p \tensor H_q(D_\chainbullet )
  \end{IEEEeqnarray*}

  Taking the long exact sequence on homology associated to
  \eqref{eq:künneth-ses-chain-complexes},
  we get
  \[
    \begin{tikzcd}[column sep = small]
      \ldots
      \ar{r}
      &
      \directsum\limits_{p+q = n} Z_p \tensor H_q (D_\chainbullet )
      \ar{r}
      &
      H_n(C_\chainbullet \chaintensor D_\chainbullet )
      \ar{r}
      \boundarymark
      &
      \directsum\limits_{p+q = n-1} B_p \tensor H_q(D_\chainbullet )
      \boundarymap{\partial}
      \\
      &
      \directsum\limits_{p+q = n-1} Z_p \chaintensor H_q(D_\chainbullet )
      \ar{r}
      &
      \ldots
    \end{tikzcd}
  \]
  A diagram chase shows that the connecting homomorphism $\partial $
  is given by
  \personal{Do diagram chase}
  \[
    \partial  = \directsum j_p \tensor \id_{H_q(D_\chainbullet )}
  ,\]
  where $j_p \colon  B_p \hookrightarrow Z_p$ is the natural inclusion.

  We may split the long exact sequence
  at $H_n(C_\chainbullet \tensor D_\chainbullet )$
  to obtain the short exact sequence
  \begin{equation}
    \label{eq:künneth-ker-coker-ses}  
    0
    \to
    \smashoperator{\directsum_{p+q =n}}
    \coker ( j_p \tensor \id_{H_q(D_\chainbullet )})
    \to 
    H_n(C_\chainbullet \chaintensor  D_\chainbullet )
    \to 
    \smashoperator{\directsum_{p+q=n-1}}
    \ker ( j_p \tensor \id_{H_q(D_\chainbullet )} )
    \to 
    0
  \end{equation}

  Now, consider the short exact sequence 
  \[
    0 \to  B_p \xrightarrow{j_p} Z_p \to H_p(C_\chainbullet ) \to 0
  ,\]
  which by the 
  \nameref{prop:tor-exact-sequence},
  and since $Z_p$ is projective,
  gives the long exact sequence
  \begin{equation}
     \label{eq:künneth-tor-exact-sequence}
     \begin{tikzcd}[column sep = small]
      &[4em]
      0
      \boundarymark
      \ar{r}
      &
      \Tor_1(H_p(C_\chainbullet ), H_q(D_\chainbullet ))
      \boundarymap{}
      \\
      B_p \tensor H_q(D_\chainbullet )
      \ar{r}{j_p \tensor \id_{H_q(D_\chainbullet )}}
      &
      Z_p \tensor H_q(D_\chainbullet )
      \ar{r}
      &
      H_p(C_\chainbullet ) \tensor H_q(D_\chainbullet )
      \ar{r}
      &
      0
    \end{tikzcd}
  \end{equation}
   Since this yields
   $\Tor_1(H_p(C_\chainbullet ), H_q(D_\chainbullet ))$
   as the kernel and
   $H_p(C_\chainbullet )\tensor H_q(D_\chainbullet )$
   as the cokernel of
   $j_p\tensor \id_{H_q(D_\chainbullet )}$,
   with taking direct sums,
   \eqref{eq:künneth-ker-coker-ses}
   becomes the desired short exact sequence
   \begin{equation}
     \label{ref:künneth-ses}
     0
     \to
     \smashoperator{\directsum_{p + q = n}}
     H_p(C_\chainbullet ) \otimes H_q (D _\chainbullet )
     \to
     H_n (C_\chainbullet \otimes D_\chainbullet )
     \to 
     \smashoperator{\directsum_{p + q = n-1}}
     \Tor_1(H_p(C_\chainbullet ), H_q(D_\chainbullet ))
     \to
     0
     .
   \end{equation}

  It remains to show the claimed splitting of the sequence.

  First, assume that $D_\chainbullet $ is also projective.
  Since $B_{p-1}$ is projective, the short exact sequence
  \eqref{eq:künneth-ses-cycles-boundaries}
  splits and we get a retraction
  $t_p \colon  C_p \to  Z_p$.
  Similarly, write $Z_q'$,  $B_q'\subset D_q$ for the cycles
  and boundaries of $D_\chainbullet $.
  Since $D_\chainbullet $ is projective, so is $B_{q-1}'$
  and in the short exact sequence
  \[
    0 \to  Z_p' \to  D_p \to B_{p-1}' \to 0
  \]
  we get a retraction $t_q' \colon  D_q \to Z_q'$.
  \begin{claim}
    The morphism
      \begin{equation*}
      \begin{array}{c c l} 
      H_n(C_\chainbullet \chaintensor D_\chainbullet )
      &
      \longrightarrow
      &
      \directsum\limits_{p+q=n}H_p(C_\chainbullet ) \tensor H_q(D_\chainbullet )
      \\
      \left[ a \chaintensor b\right]
      &
      \longmapsto
      &
      \left[t_p a\right] \tensor \left[t_q'b\right]
      \end{array}
    \end{equation*}
    is a retraction of the claimed short exact sequence.
  \end{claim}
  \begin{proof}
    It suffices to see that the first map in
    \eqref{ref:künneth-ses} is given by
    \[
      [ a ] \tensor [b] \mapsto [ a \chaintensor  b ]
    ,\]
    since then commutativity follows directly from the retraction properties.
    For this, note that we identified
    \[
      H_p(C_\chainbullet ) \tensor H_q(D_\chainbullet )
      \cong
      \coker \left( j_p \tensor \id_{H_q(D_\chainbullet )} \right) 
    \]
    with the projection
    \[
      Z_p \tensor H_q(D_\chainbullet ) \to H_p(C_\chainbullet ) \tensor H_q(D_\chainbullet )
    \]
    induced by the inclusion $Z_p \to C_p$
    and that the first map in
    \eqref{eq:künneth-ker-coker-ses}
    is induced via the long exact sequence on homology of
    \eqref{eq:künneth-ses-chain-complexes},
    and thus induced by the natural inclusion $Z_p \to  C_p$
    as well.
    Thus, we really get the desired natural form of the morphism.
  \end{proof}

  It remains to show the splitting property for general $D$.
  By
  \autoref{lm:quasi-iso-from-projective-complex},
  we can pick a projective chain complex $F_\chainbullet $
  with a quasi-isomorphism $F_\chainbullet \to D_\chainbullet$.

  This induces a morphism of short exact sequences
  \[
    \begin{tikzcd}[column sep = tiny]
      0
      \ar{r}
      &
      \smashoperator{\directsum\limits_{p+q=n}}
      H_p(C_\chainbullet ) \tensor H_q(F_\chainbullet )
      \ar{r}
      \ar{d}[swap]{\cong}
      &
      H_n(C_\chainbullet \chaintensor F_\chainbullet )
      \ar{r}
      \ar{d}{f}
      &
      \smashoperator{\directsum\limits_{p+q=n-1}}
      \Tor_1(H_p(C_\chainbullet), H_q(F_\chainbullet ))
      \ar{r}
      \ar{d}{\cong}
      &
      0
      \\
      0
      \ar{r}
      &
      \smashoperator{\directsum\limits_{p+q=n}}
      H_p(C_\chainbullet ) \tensor H_q(D_\chainbullet )
      \ar{r}
      &
      H_n(C_\chainbullet \chaintensor D_\chainbullet )
      \ar{r}
      &
      \smashoperator{\directsum\limits_{p+q=n-1}}
      \Tor_1(H_p(C_\chainbullet ), H_q(D_\chainbullet )
      \ar{r}
      &
      0
    \end{tikzcd}
  \]
  where by the 5-lemma, $f$ is an isomorphism and thus the bottom
  short exact sequence splits as well.
\end{refproof}


\begin{lemma}+
  \label{lm:quasi-iso-from-projective-complex}
  Let $C_\chainbullet $ be a chain complex.
  Then there exists a projective chain complex $P_\chainbullet $
  with a quasi-isomorphism $P_\chainbullet \to C_\chainbullet $.
\end{lemma}

\begin{proof}*
  Was left as an exercise in the lecture.
  The idea is to take a $1$-dimensional projective resolution
  of each  $H_n(D_\chainbullet )$.
\end{proof}


\begin{remark}+
  Note that for the morphism of short exact sequences
  we need to check a few naturality conditions
  (which are all trivially met):
  \begin{itemize}
    \item Functoriality of the long exact sequence on homology
      to get a morphism of the short exact sequences
      given by 
      \autoref{eq:künneth-ker-coker-ses}.
    \item Functoriality of the
      \nameref{prop:tor-exact-sequence},
      to get a morphism of the sequences given by
      \eqref{eq:künneth-tor-exact-sequence}
   \item Naturality of kernel and cokernel,
     so that the morphism of short exact sequences remains natural
     after identification of kernel and cokernel via
     \eqref{eq:künneth-tor-exact-sequence}.
  \end{itemize}
\end{remark}

\begin{remark}
  \label{rm:cellular-chain-complex-of-product-of-cw-complexes-is-tensor-product}
  Let $X$,  $Y$ be  CW-complexes and let $Y$ be locally finite.
  Then,  $X\times Y$ is again a CW-complex with its set of
  $n$-cells given by 
  \[
    I_n(X\times Y) = \coprod_{p+q = n} I_p(X) \times I_q(Y)
  .\]
  In particular,
  \[
    C_n^{\cell}(X\times Y;R)
    =
    \directsum_{p+q = n}
    C_p^{\cell}(X;R) \tensor C_q^{\cell}(Y;R)
  .\]
  By inspecting the attaching maps,
  we see that
  \[
    d(a \tensor b)
    =
    da \tensor b + (-1)^{\chaindimension{a}} \cdot a \tensor db
  \]
  and thus also
  \[
    C_\chainbullet ^{\cell}(X\times Y;R)
    \cong
    C_\chainbullet ^{\cell}(X;R) \tensor C_\chainbullet ^{\cell}(Y;R)
  \] 
\end{remark}

\begin{corollary}[Künneth for CW-complexes]
  \label{cor:künneth-for-cw-complexes}
    Let $R$ be a principal ideal domain,
    let $X$, $Y$ be  CW-complexes
    and let $Y$ be locally finite.

    Then, there exists a natural short exact sequence
    \begin{IEEEeqnarray*}{rCClCl}
      0
      &
      \to
      &
      \directsum_{p+q=n}
      &
      H_p(X;R) \tensor H_q(Y;R)
      &
      \to
      &
      H_n(X\times Y;R)
      \\
      &
      \to
      &
      \directsum_{p+q=n-1}
      &
      \Tor_1(H_p(X;R), H_q(Y;R))
      &
      \to
      &
      0
    \end{IEEEeqnarray*}
\end{corollary}
\begin{proof}*
  Apply \autoref{thm:künneth-for-chain-complexes} to the
  cellular chain complexes of $X$ and  $Y$,
  using the facts we observed in
  \autoref{rm:cellular-chain-complex-of-product-of-cw-complexes-is-tensor-product}.
\end{proof}

\begin{example}
  \begin{enumerate}[1.]
    \item Let $d\geq 1$.
      Then
      \[
        H_n(S^d;R)
        =
        \begin{cases}
          R & \text{if $n=0$ or  $n=d$} \\
          0 & \text{else}
        \end{cases}
      \]
      Moreover, $\Tor_1^R(M,R) = 0$ for all $M$,
      since $R$ is free and thus projective.
      Thus, for CW-complexes $X$, we obtain the formula
       \[
        H_n( X \times S^d ; R)
        \cong
        H_n(X ; R)
        \oplus
        H_{n-d}(X;R)
      \]
      Inductively, this yields formulas for the homology
      of $S^{d_1} \times  \dotsb \times S^{d_n};\mathbb{Z}$.
    \item Recall that
      $H_n(\mathbb{R}\mathbb{P}^2;\mathbb{Z}) = \mathbb{Z}, \mathbb{Z} /2, 0, \ldots$
      and that $\Tor_1^{\mathbb{Z}}( \mathbb{Z}/2, \mathbb{Z}/2 ) = \mathbb{Z}/2$.

      Thus we obtain
      \[
        H_n(\mathbb{R}\mathbb{P}^2 \times \mathbb{R}\mathbb{P}^2; \mathbb{Z})
        =
        \mathbb{Z}, ( \mathbb{Z}/2)^2, \mathbb{Z}/2, \mathbb{Z}/2,0,\ldots
      \] 
  \end{enumerate}
\end{example}
