%! TEX root = ../../master.tex
\lecture[]{Mo 04 Apr 2022}{The Künneth formula}

\section{Reminder on UCT}

\begin{editor}
  We revised the following notions from last semester:
  \begin{itemize}
    \item Projective modules
    \item The $\Tor$ functor and its derived long exact sequences
    \item The UCT theorem for chain complexes and spaces
  \end{itemize}
\end{editor}

\begin{exercise}
  Repeat the story with the  $\Hom$ and  $\Ext$ functors for cohomology.
\end{exercise}



\section{The Künneth theorem}

\begin{question}
  Let $X$ and  $Y$ be spaces.
  What is  $H_\chainbullet (X \times Y)$?
\end{question}

\begin{definition}
  Let $C$ and  $D$ be  $R$-chain complexes.
  Define a chain complex $C \otimes D$ by
  \[
    (C \otimes D)_n \coloneqq \bigoplus_{p + q = n} C_p \otimes D_q
  \]
  and
  \[
  d ( a \otimes b) \coloneqq  da \otimes b + (-1)^{\abs{a} } \cdot a \otimes db
  \] 
\end{definition}

\begin{theorem}[Künneth for chain complexes]
  Let $R$ be a principal ideal domain,
   $C$ a projective chain complex and  $D$ any chain complex.
   Then, there is a natural short exact sequence.
    \[
     0
     \to
     \bigoplus_{p + q = n} H_pC \otimes H_q D 
     \to
     H_n (C \otimes D)
     \to 
     \bigoplus_{p + q = n-1} \Tor_1(H_pC, H_qD)
     \to
     0
   \]
   that splits (not naturally).
\end{theorem}
