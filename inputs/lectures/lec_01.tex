%! TEX root = ../../master.tex
\lecture[]{Mo 04 Apr 2022}{The Künneth formula}

\section{Reminder on UCT}

\begin{editor}
  We revised the following notions from last semester:
  \begin{itemize}
    \item Projective modules
    \item The $\Tor$ functor and its derived long exact sequences
    \item The UCT theorem for chain complexes and spaces
  \end{itemize}

  Here, we give a brief overview over these, refer to the notes of last
  semester for more details, especially for proofs.
\end{editor}

\begin{definition}[The $\Tor$ functor]
  Let $M$ and  $N$ be  $R$-modules.
  Pick a projective resolution $P$ of $M$ and define
  \[
  \Tor_n^R(M,N) \coloneqq  H_n ( P \otimes _R N )
  .\] 
\end{definition}

\begin{lemma}
  We have
  \begin{enumerate}[h]
    \item $\Tor_n^R(M,N) \cong \Tor_n^R (N,M)$.
    \item  If  $R$ is a principal ideal domain,
      then $\Tor_n^R = 0$ for $n\geq 2$.
    \item If $M$ is a projective module,
      then  $\Tor_n^R(M, \dash) = 0$ for $n\geq 1$.
  \end{enumerate}
\end{lemma}

\begin{proposition}[$\Tor$ exact sequence]
  \label{prop:tor-exact-sequence}

  Let $0 \to  A \to  B \to  C \to 0$  be a short exact sequence
  of $R$-modules.
  Then, there is a long exact sequence of $R$-modules,
  given by
  \[
    \begin{tikzcd}[column sep = tiny]
         \ldots \ar{r} & \Tor_2(B,M) \ar{r} \ar[phantom, d, ""{coordinate, name=Z1}]& \Tor_2(C,M)
        \ar[swap,dll,
        "δ",
        rounded corners,
        to path = {
            -- ([xshift=2em]\tikztostart.east)
            |- (Z1) [near end]\tikztonodes
            -| ([xshift=-2ex]\tikztotarget.west)
        -- (\tikztotarget)}] \\
         \Tor_1(A,M) \ar{r}
         &
         \Tor_1(B,M) \ar{r}
         \ar[phantom, d, ""{coordinate, name=Z2}]
         &
         \Tor_1(C,M)
        \ar[swap,dll,
        "δ",
        rounded corners,
        to path = {
            -- ([xshift=2em]\tikztostart.east)
            |- (Z2) [near end]\tikztonodes
            -| ([xshift=-2ex]\tikztotarget.west)
        -- (\tikztotarget)}] \\
        A \otimes  M \ar{r}
        &
        B \otimes  M \ar{r}
        &
        C \otimes M \ar{r}
        &
        0
    \end{tikzcd}
  \]
\end{proposition}

\begin{theorem}[UCT for chain complexes]
  \label{thm:uct-for-chain-complexes}

  Let $R$ be a  principal ideal domain,
  $C$ a projective chain complex
  and $M$ an  $R$-module.

  Then, there is a natural short exact sequence
  \[
    \begin{tikzcd}[column sep = small, row sep = tiny]
      0 \ar{r}
      &
      H_n(C) \otimes M \ar{r}
      &
      H_n(C \otimes M) \ar{r}
      &
      \Tor_1(H_{n-1}(C), M) \ar{r}
      &
      0 \\
      &
      \left[ a \right] \otimes m \ar[mapsto]{r}
      &
      \left[ a \otimes m \right] 
    \end{tikzcd}
  \]
  Moreover, this sequence splits, but the splitting is not natural.
\end{theorem}

\begin{corollary}[UCT for spaces]
  \label{cor:uct-for-spaces}
  Let $R$ be a principal ideal domain, $X$ any space and  $M$ an  $R$-module.

  Then, there is a  short exact sequence
  \[
    0 \to  H_n(X;R) \otimes M \to  H_n(X ; M) \to \Tor_1(H_{n-1}(X;R),M) \to 0
  \] 
  Moreover, this sequence splits, but the splitting is not natural.
\end{corollary}

\begin{exercise}
  Repeat the story with the  $\Hom$ and  $\Ext$ functors for cohomology.
\end{exercise}




\section{The Künneth theorem}

In this section, we want to develop tools for computing the
homology of a product of spaces, i.e.~we want to know what
$H_\chainbullet  (X \times Y)$ is.

\begin{definition}
  \label{def:tensor-product-chain-complexes}
  Let $C_\chainbullet $ and  $D_\chainbullet $ be  $R$-chain complexes.
  We define the \vocab{tensor product of chain complexes}, denoted
  $C_\chainbullet \chaintensor  D_\chainbullet $,
  as the complex with modules
  \[
    (C_\chainbullet  \chaintensor D_\chainbullet )_n
    \coloneqq
    \directsum_{p + q = n} C_p \chaintensor D_q
  \]
  and differential
  \[
    d ( a \chaintensor b)
    \coloneqq
    da \chaintensor b + (-1)^{\chaindimension{a} } \cdot a \chaintensor db
  .\] 
\end{definition}

\begin{well-definedness}
  We have to check that the resulting complex
  does indeed satisfy $d^2 = 0$.
  The computation is straightforward:
  \begin{IEEEeqnarray*}{rCl}
    d^2 ( a \chaintensor b )
    &=&
    d ( da \chaintensor  b) + (-1)^{\chaindimension{a} } \cdot d ( a \chaintensor  db )
    \\
    &=&
    d^2a
    + (-1)^{\chaindimension{da}} da \chaintensor db
    + (-1)^{\chaindimension{a}} da \chaintensor  db
    + (-1)^{\chaindimension{a}} \cdot (-1)^{\chaindimension{a}} a \chaintensor d^2b
    \\
    &=&
    0
  \end{IEEEeqnarray*}
\end{well-definedness}

\begin{theorem}[Künneth for chain complexes]
  \label{thm:künneth-for-chain-complexes}
  Let $R$ be a principal ideal domain,
  $C_\chainbullet $ a projective chain complex and  $D_\chainbullet $ any chain complex.
  Then, there is a natural short exact sequence
  \[
     0
     \to
     \smashoperator{\directsum_{p + q = n}}
     H_p(C_\chainbullet ) \otimes H_q (D _\chainbullet )
     \to
     H_n (C_\chainbullet \otimes D_\chainbullet )
     \to 
     \smashoperator{\directsum_{p + q = n-1}}
     \Tor_1(H_p(C_\chainbullet ), H_q(D_\chainbullet ))
     \to
     0
  .\]
  Moreover, this sequence splits, but the splitting is not natural.
\end{theorem}

\begin{remark}[UCT as special case of \nameref{thm:künneth-for-chain-complexes}]
  Let $C_\chainbullet $ be a chain complex and $M$ an $R$-module. 
  Define the complex $D_\chainbullet $ as the complex
  \[
    \ldots \to  0 \to  M \to  0 \to  \ldots
  \]
  where $M$ is in degree $0$.
  Then the degreewise tensor with $M$ agrees
  with the tensor product of chain complexes with $D_\chainbullet $,
  i.e.~we have $C\chaintensor  D_\chainbullet  = C_\chainbullet \tensor M$.
  Furthermore, \autoref{thm:künneth-for-chain-complexes} specializes
  to the \nameref{thm:uct-for-chain-complexes}.
\end{remark}

\begin{refproof}{thm:künneth-for-chain-complexes}
  As usual, we denote the boundaries and cycles of $C_\chainbullet $ by
  \[
    Z_p \coloneqq \set{ a\in C_p \suchthat da = 0 },
    \qquad
    B_p \coloneqq \set{ da \suchthat  a \in C_{p+1} } 
  \]
  We have the short exact sequence
  \[
    0 \to  Z_p \xrightarrow{ι}   C_p \xrightarrow{d}  B_{p-1} \to 0
  \]
  Since $R$ is a  principal ideal domain and $C_p$ is projective,
  $Z_p$ and  $B_p$ are also projective as direct summands
  of a projective module.
\end{refproof}
