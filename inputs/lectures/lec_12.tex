%! TEX root = ../../master.tex
\lecture[]{Mo 16 Mai 2022}{Untitled}

\begin{definition}[Orientation of manifolds with boundary]
  \label{def:orientation-of-manifold-with-boundary}
  Let $M$ be a manifold with boundary.
  We say that $M$ is $R$-orientable if
  $M^0 \coloneqq  M \setminus \partial M$
  is $R$-orientable.
\end{definition}

\begin{lemma}
  Let $M$ be an $R$-orientable, $d$-dimensional manifold
  with boundary.
  Then $H_d(M, \partial M) \cong R$.
\end{lemma}
\begin{proof}
  Let $K \coloneqq M \setminus ( \partial M \times [0,1) ) \subset M_0$.
  Note that $K$ is compact.
  By the five lemma, we get that
  we have an isomorphism
  \[
    H_d(M, \partial M)
    \cong
    H_d(M, \partial M \times [0,1))
  \]
  But now using excision, we see that
  \[
    H_d(M, \partial M \times [0,1))
    \cong
    H_d(M^0, \partial M \times (0,1))
    \cong
    H_d(M^0, M_0 \K)
    \cong
    R
  \]
  by
  \autoref{thm:unique-lifts-on-compact-subsets-in-manifold-and-iso-for-connected} .
\end{proof}

\begin{remark}
  For each $x\in M^{\circ}$ consider the inclusion
  \[
    ι_x \colon H_d(M, \partial M)
    \to
    H_d(M, M\setminus \set{ x } )
    \cong
    H^d(M^{\circ}, M^{\circ} \setminus \set{ x } )
  .\]
  Then,
  \begin{enumerate}[h]
    \item Each $ι_x$ is an isomorphism
    \item Choosing a generator $[M, \partial m] \in H_d(M, \partial M)$
      is the same as choosing an orientation $(μ_x)_{x\in M^{\circ}}$.
  \end{enumerate}
  The element $[M, \partial M]$ is called the
  \vocab{fundamental class}, and we con construct
  $μ_x \coloneqq ι_x [M, \partial M]$.
\end{remark}
\todo{work this out}

\begin{remark}
  Let $M$ be a $d$-dimensional manifold with boundary.
  \begin{itemize}
    \item
      The boundary of $M$ is a $(d-1)$-dimensional manifold without boundary.
    \item If $M$ is compact, $\partial M$ is a closed manifold.
  \end{itemize}
\end{remark}

\begin{lemma}
  The connecting morphism of the pair sequence of $(M, \partial M)$
  sends $[M, \partial M]$ to some fundamental class of $\partial M$.
  That is, each orientation $[M, \partial M]$ induces an orientation
  $\partial [M, \partial M]$ on $\partial M$.
\end{lemma}

\begin{proof}
  Let $y\in \partial M$.
  Define $x \coloneqq (y, \frac{1}{2)}\in \partial M \times [0,1)$
  using the collar.
  Let
  \[
    M_1 \coloneqq M \setminus (\partial M \times [0, 1/2)
    \qquad
    M_2 \coloneqq \partial M \times [0, 1/2]
  .\]
  Then define
  \[
    M_0 \coloneqq M_1 \cap M_2 = \partial M \times \set{ 1/2 } 
  \]
  and consider the diagram
  \[
    \begin{tikzcd}
      (M, \partial M)
      \ar[swap]{d}{}
      \ar{r}{}
      &
      M_1 \cup _{M_0} (M_2, \partial M)
      \ar{d}{}
      \\
      (M, M\setminus \set{ x } )
      \ar[swap]{r}{}
      &
      (M_1, M_1 \setminus \set{ x } ) \cup _{M_0, M_0 \setminus \set{ x } } (M_2, M_2 \setminus \set{ x } )
    \end{tikzcd}
  .\]
  Now, we get a morphism of Mayer-Vietoris sequences
  \[
    \begin{tikzcd}
      H_d(M, \partial M)
      \ar{r}
      &
      H_{d-1}(M_0)
      \ar{rr}{}
      &
      &
      H_{d-1}(M_1) \oplus H_{d-1}(M_2, \partial M)
      \\
      H_d(M, M \setminus \set{ x })
      \ar{r}
      &
      H_{d-1}(M_0, M_0\setminus \set{ x } )
      \ar{rr}{}
      &
      &
      H_{d-1}(M_1, M_1 \setminus \set{ x } )
      \oplus
      H_{d-1}(M_2, M_2 \setminus \set{ x } )
    \end{tikzcd}
  \]
  We see that for $i = 1$, $2$ that $M_i \\\set{ x } \to M_i$
  is a homotopy equivalence, and thus
  $H_\chainbullet (M_i, M_i \setminus \set{ x })  = 0$.
  Also note that
  $H_d(M, \partial M) \to H_d(M, M\setminus \set{ x } )$
  is an isomorphism by 8.7\todoref.

  Now consider the map $\varphi \colon M \to  M$
  by interval resealing, squashing $\partial M \times [0, 1/2]$
  to $\partial M$ and stretching $\partial M \times [1/2,]$.
  This gives a morphism of excisive triads
  \[
    \begin{tikzcd}
      (M, \partial M)
      \ar[swap]{d}{}
      \ar{r}{}
      &
      M_1 \cup _{M_0} (M_2, \partial M)
      \ar{d}{}
      \\
      (M, \partial M)
      \ar[swap]{r}{}
      &
      M \cup _{\partial M} (\partial M, \partial M)
    \end{tikzcd}
  \]
  We note that $M_0 \to \partial M$ is just an isomorphism,
  and that $(\partial M, \partial M)$ is trivial,
  so that the Mayer-Vietoris sequence will just be
  the long exact sequence of pairs.

  Moreover, $\varphi $ restricts to some morphism
  $\varphi \colon  (M_0, M_0 \setminus \set{ x } )
  \to (\partial M, \partial M \setminus \set{ y } $,
  giving us another induced map on Mayer-Vietoris-sequences.
  Now we see that the morphism
  \[
    H_d(M, \partial M) \xrightarrow{\partial }
    H_{d-1}(\partial M)
    \xrightarrow{f_y}
    H_{d-1}(\partial M, \partial M \setminus \set{ y } )
  \]
  is an isomorphism.
  Thus, $f_y(\partial [M, \partial M])$ is a generator for each $y\in \partial M$,
  but this is just the statement that $\partial [M, \partial M]$
  is a fundamental class of $\partial M$.
\end{proof}

\begin{editor}
  Just a placeholder to fix numbering for now.
  \missingfigure{picture}
\end{editor}

\begin{definition}[Bordism]
  Let $M$ and $M'$ be two closed, $d$-dimensional,
  \emph{oriented} manifolds.
  A \vocab{bordism} $W\colon M \to  M'$
  is a $(d+1)$-dimensional, compact, \emph{oriented}
  manifold with
  \[
    \partial W \cong M \sqcup (-M')
  .\],
  where $-M'$ denotes $M'$ with reversed orientation,
  and $\partial W$ carries the induced orientation of $W$.

  We say that $M$ and $M'$ are \vocab{bordant},
  if there exists a bordism $W\colon  M \to  M'$.
\end{definition}

\begin{lemma}+
  Bordism is an equivalence relation of closed, oriented manifolds.
\end{lemma}
\begin{proof}
  For reflexivity, notice that we can take
  the projection $M \times [0,1] \colon M \to  M$
  and orient it in a natural way.

  For symmetric, if $W\colon  M \to  M'$,
  reversing the orientation of $W$ yields
  a bordism $-W \colon M' \to  M$.

  For transitivity, if $W\colon M \to  M'$
  and $W' \colon M' \to  M''$ are two bordisms,
  then $W' \circ W \coloneqq W \cup _{M'} W'$
  is a bordism $M \to  M''$.
\end{proof}

\begin{lemma}
  Denote
  \[
    \Omega_d \coloneqq \faktor{\set{ \text{$d$-dim, oriented manifolds} } }{\text{bordism}}
  .\]
  Then
  \begin{enumerate}[h]
    \item $\Omega_d$ is an abelian group by setting
      \[
        [M] + [N] = [M \sqcup N]
      \],
      called the \vocab{$d$-dimensional bordism group}. 
    \item $\Omega \coloneqq  \directsum_{d\geq 0} \Omega_d$
      is a graded-commutative ring, called the
      \vocab{bordism ring}.
  \end{enumerate}
\end{lemma}
\todo{change environment}
\todo{note on set}
\todo{stuff on unoriented version}

\begin{theorem}[Poincarè-Lefschetz]
  Let $M$ be a $d$-dimensional, compact, orientable manifold with boundary.
  Then,
  \[
    \dash \cap [M, \partial M]\colon H^p(M, \partial M) \to H_{d-p}(M)
  \]
  is an isomorphism.
\end{theorem}

\begin{proof}
  The family $K_n \coloneqq M \setminus (\partial M \times [0, 1/n))_{n\geq 1}$
  is cofinal in the system of compact subsets of $M^{\circ}$.
  Thus,
  \[
    H_{cs}^p(M^{\circ})
    \cong
    \lim_{n\in \mathbb{N}} H^p(M^{\circ}, M^{\circ} \setminus K_n)
    \cong
    \lim_{n\in \mathbb{N}} H^p(M, M\setminus K_n)
  .\]
  Also note that we have a system of isomorphisms
  \[
    H^p(M, M\setminus K_1) \to H^p(M, M\setminus K_2) \to \ldots
  \]
  Thus, we get an induced isomorphism
  \[
    H_{CS}^p(M^{\circ})
    \xrightarrow{\varphi }
    H^p(M, \partial M)
  .\]
  Note that this already yields that we have an abstract
  isomorphism, but not necessarily given by cupping.

  Now the fundamental class $[M, \partial M]$ gives us
  generators $μ_n \in H_d(M, M\setminus K_n)
  \cong H_d (M^{\circ}, M^{\circ} \setminus K_n)$
  and we know that for all $α\in H^p(M, M\setminus K_n)$,
  $α \cap μ^n = \varphi _n(α) \cap [M, \partial M]$.
  \todo{check this, adjunction formula!}
  As poincare duality was constructed by $\lim (- \cap μ_n)$,
  we deduce that there is some commutative diagram
  \[
    \begin{tikzcd}
      H_{cs}^p (M^{\circ})
      \ar[swap]{d}{PD}
      \ar{r}{\varphi }
      &
      H^p(M, \partial M)
      \ar{d}{- \cap [M, \partial M]}
      \\
      H_{d-p}(M^{\circ})
      \ar[swap]{r}{}
      &
      H_{d-p}(M)
    \end{tikzcd}
  .\]
\end{proof}

\begin{theorem}[Poincarè-Lefschetz 2]
  Let $M$ be a compact, orientable, $d$-dimensional manifold
  with boundary.
  Then,
  \[
    - \cap [M, \partial M] \colon H^p(M) \to H_{d-p}(M, \partial M)
  \]
  is an isomorphism.
\end{theorem}

\begin{proof}
  Very generally, the following diagrams commute
  for each pair $(X,A)$.
  \[
    \begin{tikzcd}
      H^k(X) \tensor H_{n+k}(X,A)
      \ar[swap]{d}{\cap }
      \ar{r}{(-1)^k \cdot (i^* \tensor \partial )}
      &
      H^k(A) \tensor H^{n+k-1}(A)
      \ar{d}{\cap }
      \\
      H_n(X,A)
      \ar[swap]{r}{\partial }
      &
      H_{n-1}(A)
    \end{tikzcd}
  \]
  This follows using $\partial (\varphi  \cap σ) = (-1)^{\abs{\varphi } }(\varphi  \cap \partial σ - δ \varphi  \cap σ)$.
  \todo{check this}.
  \begin{tikzcd}
    H^k(A) \tensor H_{n+k-1}(A)
    &
    H^k(A) \tensor H_{n+k}(X,A)
    \ar{l}{1 \tensor  \partial }
    \ar{r}{δ \tensor 1}
    &
    H^{k+1}(X,A) \tensor H_{n+k}(X,A)
    \ar{d}{\cap }
    \\
    H_{n-1}(A)
    \ar{rr}{ι_*}
    &
    &
    H_{n-1}(X)
  \end{tikzcd}
  using the same boundary formula.
\end{proof}
