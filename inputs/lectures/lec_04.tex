%! TEX root = ../../master.tex
\lecture[$(p,q)$-shuffles. The maps in Eilenberg-Zilber, the Alexander-Whitney map. Cross product definitions. Topological manifolds.]{Mi 13 Apr 2022}{Manifolds and orientations}

\begin{example}
  We want to give two concrete examples of chain maps
  appearing in \autoref{thm:eilenberg-zilber}.

  If $p$ and  $q$ are non-negative integers,
  a  \vocab{$(p,q)$-shuffle} is a pair of disjoint sets of integers
  $1 \leq \mu_1 < μ_2 < \dotsb < \mu_p \leq p + q$
  and
  $1 \leq \nu_1 < \nu_2 < \dotsb < \nu_q \leq p + q$.
  Let $\sgn(μ,ν)$ be the sign of the permutation given by
   $(\mu_1, \dotsc, \mu_p, \nu_1, \dotsc, \nu_q)$.

   Define a linear map $\eta^{\mu}\colon \Delta^{p+q} \to  \Delta^p$ 
   by $\eta^{\mu}(e_i) = e_j$ if $\mu_j \leq i < \mu_{j+1}$
   (where by convention we have $\mu_0 = 0$, $\mu_{p+1} = p + q + 1$ ).

   Define homomorphisms
   %
   \begin{equation*}
     \nabla_{p,q}: \left| \begin{array}{c c l} 
     C^{\sing}_p (X;R) \tensor _R C_q^{\sing}(Y;R) & \longrightarrow & C^{\sing}_{p+q}(X\times Y;R) \\
     σ \tensor τ & \longmapsto &
     \sum\limits_{\substack{(\mu, \nu) \\ \text{$(p,q)$-shuffle}} }
     \sgn(\mu, \nu) (σ \circ \eta^{\mu}, τ \circ  \eta^{\nu})
     \end{array} \right.
   \end{equation*}
   %
   Then
   \[
     (\Csing(X;R) \tensor _R \Csing(Y;R))_n
     =
     \directsum_{p+q=n} \C_p(X;R) \tensor _R C_q(Y;R)
     \xrightarrow{\oplus \nabla_{p,q}} 
     C_{p+q}(X\times Y;R)
   \]
   is a chain homotopy equivalence, as one checks.
   By checking that this induces our isomorphism on $H_0$,
   one also sees that this has to be the map in \nameref{thm:eilenberg-zilber}.

   An inverse is given as follows:

   Define
     \begin{equation*}
     \nu_{p,q}: \left| \begin{array}{c c l} 
     \Delta^p& \longrightarrow & \Delta^{p+q} \\
     e_i & \longmapsto & e_i
     \end{array} \right.
   \end{equation*}
   and
     \begin{equation*}
     h_{p,q}: \left| \begin{array}{c c l} 
     \Delta^q & \longrightarrow & \Delta^{p+q} \\
     e_i & \longmapsto &  e_{p+i}
     \end{array} \right.
     .
   \end{equation*}
   The \vocab{Alexander-Whitney} map is then defined as
     \begin{equation*}
     A: \left| \begin{array}{c c l} 
     \Csing(X\times Y;R) & \longrightarrow & \Csing(X;R) \tensor \Csing(Y;R) \\
     σ & \longmapsto &
     \directsum\limits_{p+q=n}
     (\pr_X \circ σ \circ \nu_{p,q}) \tensor (\pr_Y \circ σ \circ h_{p,q})
     \end{array} \right.
     .
   \end{equation*}
   Again, check that this is indeed a chain map and induces the
   identity in the zero dimension.
   Thus it is the inverse map that appears in \nameref{thm:eilenberg-zilber}.
\end{example}

Recall that last semester, we gave another definition of the cup product:

\begin{definition}+[Last years version of cup and cross product]
  \label{def:cup-and-cross-product-last-term}
  For singular cohomology, define the cup and cross product by
  \[
    (\varphi  \cup \psi ) (σ)
    =
    \varphi (\frestriction{σ}{[0,\dotsc,p]})
    \cdot
    \psi ( \frestriction{σ}{[p, \dotsc, p+q]}
  \]
  and
  \begin{IEEEeqnarray*}{RrCl}
    \times \colon
    &
    H^p(X) \tensor H^q(Y)
    &
    \xrightarrow{\pr_X^* \tensor \pr_Y^*}
    &
    H^p(X\times Y) \tensor H^q(X\times Y)
    \\
    &
    &
    \xrightarrow{\cup } 
    &
    H^{p+q}(X\times Y)
  \end{IEEEeqnarray*}
\end{definition}

\begin{proposition}+
  \label{prop:cup-product-definitions-agree}
  The definitions of the cross product given in
  \autoref{def:cohomological-cross-product}
  and
  \autoref{def:cup-and-cross-product-last-term}
  agree.
\end{proposition}

\begin{proof}
  Examining the first two maps of
  \autoref{def:cohomological-cross-product},
  we get that
  \begin{IEEEeqnarray*}{rCl}
    H_p(\Csing*(X,A)) \tensor H_q(\Csing*(Y,B))
    &
    \to
    &
    H_{p+q}(\Csing*(X,A) \tensor \Csing*(Y,B))
    \\
    &
    \to
    &
    H_{p+q}((\Csing(X,A) \tensor \Csing(Y,B))^{\chainbullet})
  \end{IEEEeqnarray*}
  is given by
  \begin{equation}
    \label{eq:cross-product-definition-first-two-maps}
    [\varphi ] \tensor [\psi ]
    \mapsto
    [ σ \tensor τ \mapsto
    \begin{cases}
      0 & \text{$\abs{σ} \neq p $ or $\abs{τ} \neq q $}
      \\
      \psi (σ) \psi (τ)  & \text{else}
    \end{cases}
    .
  \end{equation}
  Now apply the Alexander-Whitney map to obtain
  \[
    \left[
    σ
    \mapsto
    \left( \pr_X \circ \frestriction{σ}{[0,\dotsc,p]} \right)
    \cdot
    \left( \pr_Y \circ \frestriction{σ}{[p,\dotsc,p+q]} \right)
    \right]
    =
    \pr_X^* \varphi \cdot \pr_Y^* \psi
  ,\]
  which is precisely the map given in
  \autoref{def:cup-and-cross-product-last-term}.
  Note that the direct sum present in the Alexander-Whitney map
  disappeared, as all summands except $(p,q)$ vanish
  as seen in
  \eqref{eq:cross-product-definition-first-two-maps}.
\end{proof}



\section{Manifolds and orientations}

\begin{definition}[Manifold]
  \label{def:topological-manifold}
  A \vocab{$d$-dimensional (topological) manifold} $M$
  is a second countable Hausdorff space
  that is locally homeomorphic to $\mathbb{R}^d$.

  That is, for each $x\in M$, there exists an open neighborhood $U \ni x$
  with  $U \cong \mathbb{R}^d$.
\end{definition}

\begin{oral}
  As each open subset of $\mathbb{R}^d$ contains an open ball,
  which is itself homeomorphic to $\mathbb{R}^d$,
  one can also relax the definition and require that each
  point has an open neighborhood homeomorphic to some
  open subset of $\mathbb{R}^d$.
\end{oral}

\begin{oral}[Smooth manifolds]
  There is also the notion of a \vocab{smooth manifold},
  which you actually might have encountered first.

  In this case, one calls the homeomorphism of open neighborhoods
  of  $x$ to  $\mathbb{R}^d$ \vocab{charts} and requires
  the composition of two charts (as a map $\mathbb{R}^d -> \mathbb{R}^d$)
  to be differentiable.
\end{oral}

\begin{example}
  The spaces $S^d$,  $R^d$,  $T^d \coloneqq \prod_{i=1}^d S^1$, $\Sigma_g$
  are all topological (and in fact smooth) manifolds.

  We will also see that $\mathbb{R}\mathbb{P}^d$ and $\mathbb{C}\mathbb{P}^d$
  are manifolds.
\end{example}

In the following, we  want to study (co)homology of manifolds.

\begin{lemma}
  \label{lm:relative-homology-around-point-in-manifold}
  Let $M$ be a $d$-dimensional manifold,
  $R$ be a ring
  and  $N$ be an  $R$-module.

  For all  $x\in M$, we have
  \[
    H_i(M, M \setminus \set{ x } ;N ) \cong \begin{cases}
      N & i = d \\
      0 & \text{else}
    \end{cases}
  .\]
\end{lemma}

\begin{proof}
  Since $M$ is a manifold,
  there exists an open neighborhood $U$ of  $x$ with a homeomorphism
   $(U,x) \cong (\mathbb{R}^d, 0)$.
   Hence by excision, we get that
   \begin{IEEEeqnarray*}{rCl}
     H_i(M, M \setminus \set{ x } ; N)
     &
     \cong
     &
     H_i(U, U \setminus \set{ x } ;N)
     \\
     &
     \cong
     &
     H_i (\mathbb{R}^d , \mathbb{R}^d \setminus \set{ 0 } ;N)
     \\
     &
     \cong
     &
     H_i(D^d, S^{d-1};N)
     \\
     &
     \cong
     &
     \begin{cases}
       N & i = d \\
       0 & \text{else}
     \end{cases}
     .
     \qedhere
   \end{IEEEeqnarray*}
\end{proof}

Any element $A\in \GL_n(\mathbb{R})$ induces an automorphism of
$H_n(\mathbb{R}^n, \mathbb{R}^n \setminus \set{ 0 }; \mathbb{Z}) \cong \mathbb{Z} $.
It is the identity if and only if $\det (A) > 0$.
So picking a generator of
$H_n(\mathbb{R}^n, \mathbb{R}^n \setminus \set{ 0 }; \mathbb{Z})$
is like picking an orientation of $\mathbb{R}^n$.

\begin{definition}[Orientation]
  \label{def:orientation}
  Let $M$ be a  $d$-dimensional manifold and $R$ a ring.

  An \vocab{$R$-orientation} of $M$ is a choice of generators 
  $\mu_x \in H_d(M, M \setminus \set{ x };R) \cong R $ 
  for each $x\in M$,
  such that for all $x\in M$,
  there exists a neighborhood $U$ of $x$
  and an element $\mu_U \in H_d(M, M \setminus U ; R)$ 
  such that for all $y\in U$, the map
  \[
    H_d(M, M \setminus U; R) \to H_d( M, M\setminus \set{ y } ;R)
  \]
  sends $\mu_X$ to  $\mu_y$.

  $M$ is called \vocab{$R$-orientable} if an $R$-orientation exists
  and  \vocab{$R$-oriented} if such an orientation is chosen.
  If we omit $R$, implicitly  $\mathbb{Z}$ is meant,
  so we will refer to just \enquote{an oriented manifold}.
\end{definition}

\begin{example}
  For $x\in \mathbb{R}^d$ let $h_x$ denote translation by $x$.
  Pick a generator
  $\mu_0 \in H_d(\mathbb{R}^d , \mathbb{R}^d \setminus \set{ 0 };\mathbb{Z})$
  and define
  $\mu_x \coloneqq (h_x)_*(\mu_0)$.
  This defines an orientation of $\mathbb{R}^d$.
\end{example}

\begin{lemma}
  \label{lm:each-manifold-is-uniquely-orientable-in-z-2}
  Every manifold has a unique $\mathbb{Z}/2$-orientation.
\end{lemma}

\begin{proof}
  Since $\mathbb{Z}/2$ has a unique generator,
  there only is one choice for $\mu_x$ for each  $x$,
  hence the orientation is unique.
  It remains to show that the only choice of $\mu_x$ is in fact
  an orientation.

  For $x\in M$, there exists a neighborhood $V$ with a homeomorphism
  $(V,x) \cong (\mathbb{R}^d, 0)$.
  Let $U\subset V$ be the image of the open unit ball.
  Then
  \[
    H_d(M, M\setminus U;\mathbb{Z}/2)
    \to
    H_d(M, M\setminus \set{ y }; \mathbb{Z}/2 )
  \]
  is an isomorphism for all $y \in U$.
  Hence the generator $\mu_X$ of $H_d(M, M \setminus U ; \mathbb{Z}/2)$
  restricted to $\mu_y$ as required.
\end{proof}

\begin{proposition}
  \label{prop:free-transitive-action-of-units-on-orientations-for-connected-manifolds}
  Let $R$ be a ring,
  $M$ a connected, $d$-dimensional, $R$-orientable manifold.
  Then the units  $R^{\times}$ of $R$ act freely and transitively
  on the set of $R$-orientations.
\end{proposition}

\begin{proof}
  The action of $r\in R^{\times}$ is given by
  sending an orientation $\set{ \mu_x }_{x\in M} $
  to $\set{ r\cdot \mu_x }_{x\in M} $.
  By replacing $\mu_U$ with $r\cdot \mu_U$,
  we see that this is also an orientation
  and it is easy to check that this is indeed an action.

  The action is obviously free and transitive on the
  generators of $H_d(M, M\setminus \set{ x }; R) \cong R$ 
  for every $x$.

  Hence the action on $\set{ \mu_x } $ is free.

  To see that the action is transitive,
  we have to show that two orientations agree when they
  agree for a single $x$.

  For this, note that
  \autoref{lm:orientation-can-locally-be-given-as-isomorphism}
  in particular gives us that for each point $x$,
  we can find a neighborhood $U$ of $x$ such that
  if two orientations $\set{ μ_x }$ and  $\set{ ν_x } $ coincide on any
  $y\in U$, then they already agree on $U$.

  Now, let  $\set{ μ_x } $ and $\set{ ν_x } $ be any two orientations.
  Considering the set  $U \coloneqq \set{ x\in X \suchthat μ_x = ν_x } $,
  by the prior statement, we see that $U$ is open.
  Now consider  $\overline{U}\subset X$, if $\overline{U}=X$,
  then $U$ is also closed and by connectedness we deduce $U=X$
  and hence $ \set{ μ_x }  = \set{ ν_x } $ as required.

  If not, pick a point $y \in \partial U$ and a neighborhood
  $V$ of $y$ with above property.
  But then as $V\cap U \neq \emptyset$, there is a point in
  $V$ where $\set{ μ_x } $ and $\set{ ν_x } $ coincide,
  so that they also coincide on $y$, but $y\not\in U$, \contra.
\end{proof}

\begin{lemma}
  \label{lm:orientation-can-locally-be-given-as-isomorphism}
  Let $M$ be a $d$-dimensional manifold.
  Let $\set{ \mu_x } $ be an $R$-orientation.
  Then, for every  $x\in M$,
  there exists a neighborhood $U$ and
  $\mu_U \in H_d(M, M \setminus U;R)$ 
  such that
  \[
    H_d(M, M\setminus U;R) \to H_d(M, M\setminus \set{ y } ;R)
  \]
  is an isomorphism sending $\mu_U$ by  $\mu_y$.
\end{lemma}

\begin{proof}
  Let $V$ be any neighborhood satisfying
  the restriction condition for $x$, i.e.~$\mu_V$ exists.
  Then there is a neighborhood of $x$ in $V$ which is homeomorphic
  to $\mathbb{R}^d$.
  Then
  \[
    H_d(M, M\setminus U;R \to H_d(M, M\setminus \set{ y } ;R)
  \]
  is an isomorphism for all $y$ and the image of  $\mu_V$
  in  $H_d(M, M\setminus U;R)$ is $\mu_U$.
\end{proof}

\begin{oral}
  Note that for a manifold $M$ and a point  $x\in M$,
  we can actually find an open neighborhood $x\in U \cong \mathbb{R}^d$
  contained in any given neighborhood $V$ of  $x$.
\end{oral}

\begin{example}
  Since $\mathbb{Z}^{\times} = \set{ \pm 1 } $,
  every connected manifold has either $0$ or  $2$ orientations.
\end{example}

In the following,
we want to show that every compact, connected,
$R$-orientable manifold $M$ satisfies
$H_d(M;R) \cong R$ and that a generator corresponds
to an orientation.
We need to establish a few things for this.

\begin{lemma}
  \label{lm:top-homology-of-manifold-relative-to-cocompact-subset}
  Let $M$ be a $d$-dimensional manifold, $N$ an  $R$-module
  and  $K\subset M$ be compact.
  Then, the following holds
  \begin{enumerate}[h]
    \item $H_i(M, M\setminus K ; N) = 0$ for $i>d$.
    \item Let $u\in H_d(M, M \setminus K;N)$ and denote
      $i_x\colon (M, M \setminus K) \to  (M, M \setminus \set{ x } )$
      for the inclusion.
      If $(i_x)_* u = 0$ for all  $x$, then $u=0$.
  \end{enumerate}
\end{lemma}

\begin{refproof}{lm:top-homology-of-manifold-relative-to-cocompact-subset}
  \begin{claim}
    The lemma holds for $M = \mathbb{R}^d$ and convex $K$.
  \end{claim}
  \begin{proof}
    First, assume that $M = \mathbb{R}^d$ and $K$ is convex.
    By translation and scaling, we can assume that
    $0\in K\subset \mathring{D}^d$.

    Now, consider the homotopy
      \begin{equation*}
      H: \left| \begin{array}{c c l} 
        (D^d \setminus \set{ 0 } ) \times [0,1]
        & \longrightarrow &
        (D^d \setminus \set{ 0 } ) \\
        (x,t) & \longmapsto &  tx + (1-t) \frac{x}{\norm{x}}
      \end{array} \right.
      .
    \end{equation*}
    Since $K$ is convex, this also induces a homotopy
    $(D^d \setminus K) \times [0,1] \to  (D^d \setminus K)$.
    It follows that
    \[
      (D^d, D^d \setminus K) \to (D^d, D^d \setminus \set{ 0 } )
    \]
    is a homotopy equivalence,
    as both are homotopy equivalent to $(D^d, S^{d-1})$.
    Thus, by excision we deduce that
    \[
      H_n(\mathbb{R}^d, \mathbb{R}^d \setminus K;N)
      \xrightarrow{(i_0)_*} 
      H_n(\mathbb{R}^d, \mathbb{R}^d \setminus \set{ 0 } ;N)
    \]
    is an isomorphism.
    The claims follow easily.
  \end{proof}
\end{refproof}
