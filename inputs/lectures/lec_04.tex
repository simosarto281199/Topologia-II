%! TEX root = ../../master.tex
\lecture[]{Mi 13 Apr 2022}{Untitled}

\begin{example}
  We want to give two concrete examples of chain maps
  appearing in \autoref{thm:eilenberg-zilber}.

  If $p$ and  $q$ are non-negative integers,
  a  \vocab{$(p,q)$-shuffle} is a pair of disjoint sets of integers
  $1 \leq \mu_1 < μ_2 < \dotsb < \mu_p \leq p + q$
  and
  $1 \leq \nu_1 < \nu_2 < \dotsb < \nu_q \leq p + q$.
  Let $\sgn(μ,ν)$ be the sign of the permutation given by
   $(\mu_1, \dotsc, \mu_p, \nu_1, \dotsc, \nu_q)$.

   Define a linear map $\eta^{\mu}\colon \Delta^{p+q} \to  \Delta^p$ 
   by $\eta^{\mu}(e_i) = e_j$ if $\mu_j \leq i < \mu_{j+1}$
   (where by convention we have $\mu_0 = 0$, $\mu_{p+1} = p + q + 1$ ).

   Define homomorphisms
   %
   \begin{equation*}
     \nabla_{p,q}: \left| \begin{array}{c c l} 
     C^{\sing}_p (X;R) \tensor _R C_q^{\sing}(Y;R) & \longrightarrow & C^{\sing}_{p+q}(X\times Y;R) \\
     σ \tensor τ & \longmapsto &
     \sum\limits_{\substack{(\mu, \nu) \\ \text{$(p,q)$-shuffle}} }
     \sgn(\mu, \nu) (σ \circ \eta^{\mu}, τ \circ  \eta^{\nu})
     \end{array} \right.
   \end{equation*}
   %
   Then
   \[
     (\Csing(X;R) \tensor _R \Csing(Y;R))_n
     =
     \directsum_{p+q=n} \C_p(X;R) \tensor _R C_q(Y;R)
     \xrightarrow{\oplus \nabla_{p,q}} 
     C_{p+q}(X\times Y;R)
   \]
   is a chain homotopy equivalence, as one checks.
   By checking that this induces our isomorphism on $H_0$,
   one also sees that this has to be the map in \nameref{thm:eilenberg-zilber}.

   An inverse is given as follows:

   Define
     \begin{equation*}
     \nu_{p,q}: \left| \begin{array}{c c l} 
     \Delta^p& \longrightarrow & \Delta^{p+q} \\
     e_i & \longmapsto & e_i
     \end{array} \right.
   \end{equation*}
   and
     \begin{equation*}
     h_{p,q}: \left| \begin{array}{c c l} 
     \Delta^q & \longrightarrow & \Delta^{p+q} \\
     e_i & \longmapsto &  e_{p+i}
     \end{array} \right.
     .
   \end{equation*}
   The \vocab{Alexander-Whitney} map is then defined as
     \begin{equation*}
     A: \left| \begin{array}{c c l} 
     \Csing(X\times Y;R) & \longrightarrow & \Csing(X;R) \tensor \Csing(Y;R) \\
     σ & \longmapsto &
     \directsum\limits_{p+q=n}
     (\pr_X \circ σ \circ \nu_{p,q}) \tensor (\pr_Y \circ σ \circ h_{p,q})
     \end{array} \right.
     .
   \end{equation*}
   Again, check that this is indeed a chain map and induces the
   identity in the zero dimension.
   Thus it is the inverse map that appears in \nameref{thm:eilenberg-zilber}.
\end{example}
