%! TEX root = ../../master.tex
\lecture[]{Mo 25 Apr 2022}{Untitled}



\begin{remark}
  As
  $H_d(M, M \setminus \set{ x }; R)
  \cong
  H_d(M, M \setminus \set{ x } ; \mathbb{Z}) \tensor _{\mathbb{Z}} R$
  by the \nameref{cor:uct-for-spaces},
  a generator on the left side corresponds to
  $e\tensor r$ with $r\in R^{\times}$ and a generator
  $e\in H_d(M, M \setminus \set{ x } ; \mathbb{Z})$
  .

  Thus, for a $\mathbb{Z}$-orientation $\set{ \mu_x } $ and $r\in R^{\times}$,
  we get an $R$-orientation  $\set{ \mu_x \tensor r} $.

  Note that that $\set{ \mu_x \tensor r } $ is in fact an orientation,
  as we have the commutative diagram
  \[
    \begin{tikzcd}
      H_d(M, M \setminus U; \mathbb{Z}) \ar[swap]{d}{} \ar{r}{} & H_d(M, M\setminus \set{ y } ; \mathbb{Z}) \tensor R \ar{d}{} \\
      H_d(M , M \setminus U; R) \ar[swap]{r}{} & H_d(M, M\setminus \set{ y } ; R)
    \end{tikzcd}
  \]
  which shows the local compatibility of the generators.
\end{remark}



\subsection{The mapping degree}

\begin{definition}[Mapping degree]
  \label{def:mapping-degree-oriented-closed-connected-manifolds}
  Let $M$ and $N$ be oriented, connected, closed,
  $d$-dimensional manifolds and $f\colon M\to N$.
  The \vocab{mapping degree} $\deg(f)\in \mathbb{Z}$
  is defined by as the unique integer such that
  \[
    f_*[M] = \deg(f)\cdot f_*[N]
  .\] 
\end{definition}
\begin{well-definedness}
  Note that by
  \autoref{thm:fundamental-class-exists-and-top-homology-is-r-iff-orientable}
  we get that $f_*$ is a map $\mathbb{Z}\to \mathbb{Z}$
  and $[N]$ is a generator.
\end{well-definedness}

\begin{notation}+
  Let $M$ be an oriented, connected,
  closed, $d$-dimensional manifold.
  As $M$ has exactly two orientations by
  \autoref{ex:connected-manifold-has-zero-or-two-orientations},
  we may denote $\overline{M}$ to mean $M$ with the unique
  other orientation.
\end{notation}

\begin{lemma}
  Let $M$, $N$ and $P$ be oriented, connected, closed,
  $d$-di\-men\-sion\-al manifolds.
  Let $f\colon M\to N$ and $g\colon  N \to P$ be maps.
  Then, the following hold:
  \begin{enumerate}[h]
    \item $\deg(f\colon M\to N) = \deg(f\colon \overline{M}\to \overline{N}) = -\deg(f\colon \overline{M}\to N)$.
    \item The mapping degree is multiplicative,
      that is,
      \[
        \deg(M\xrightarrow{f} N\xrightarrow{g} P)
        =
        \deg(f)\cdot \deg(g)
      \]
    \item If $\deg(f) \neq 0$, then $f$ is surjective.
  \end{enumerate}
\end{lemma}

\begin{proof}
  For $1)$ it suffices to note that $[\overline{M}] = - [M]$.

  For $2)$, by functoriality, we compute that
  \begin{IEEEeqnarray*}{rCl}
    \deg(g \circ f) [P]
    &
    =
    &
    (g \circ f)_*[M]
    =
    g_*(f_*[M])
    \\
    &
    =
    &
    g_*(\deg(f)[N])
    =
    \deg(f)g_*[N]
    =
    \deg(f)\deg(g)[P]
  \end{IEEEeqnarray*}
  and thus $\deg(g\circ f) = \deg(f)\deg(g)$ as claimed.

  For $3)$, assume that  $y\not\in \im f$,
  we will show that $\deg(f) = 0$.
  Consider the composition
  \[
    \begin{array}{c c c c l}
      H_d(M;\mathbb{Z})
      &
      \longrightarrow
      &
      H_d(N,\mathbb{Z})
      &
      \xrightarrow{\cong}
      &
      H_d(N,N\setminus \set{ y } ;\mathbb{Z})
      \\
      \left[M\right]
      &
      \longmapsto
      &
      \deg(f)\left[N\right]
      &
      \longmapsto
      &
      \deg(f)\cdot \mu_y=0
    \end{array}
  .\]
  But since $y\not\in \im f$, the composition is trivial,
  but as the second map is a surjection, we deduce that already
  $f_*[M]$ is trivial and thus $\deg(f) = 0$.
\end{proof}

\begin{oral}
  One immediately gets that homotopy equivalences between
  oriented, closed, connected, $d$-dimensional manifolds
  are surjective.
\end{oral}

\begin{proposition}
  \label{prop:degree-is-fibre-counted-with-signs}
  Let $M$ and $N$ be oriented, connected, closed,
  $d$-di\-men\-sion\-al manifolds
  and $f\colon M \to  N$ a map.
  Let  $x$ be a point in $N$ with a neighborhood $U$ such that
  $f^{-1}(U) = \coprod_{i \in I}U_i$
  such that each $U_i$ is mapped homeomorphically onto  $U$.
  Then, for each $y_i \coloneqq f^{-1}(x) \cap U_i$,
  define $ε_i\in \set{ +1, -1 } $ to be the sign such that
  \[
      H_d(M, M \setminus \set{ y_i } )
      \xleftarrow{\cong}
      H_d(U_i, U_i \setminus \set{ y_i } )
      \mathrel{\mathop{\rightarrow}^{\cong}_{f}}
      H_d(U, U\setminus \set{ x } )
      \xrightarrow{\cong}
      H_d(N, N\setminus \set{ x } )
  \]
  sends $μ_{y_i}$ to to $ε_i μ_x$.
  Then,
  \[
    \deg(f) = \sum_{i \in I} ε_{i}
  .\] 
\end{proposition}

\begin{remark}+
  One often phrases the statement of 
  \autoref{prop:degree-is-fibre-counted-with-signs}
  as
  \enquote{$\deg(f) = \abs{f^{-1}(x)}$ with signs}.
\end{remark}

\begin{remark}+
  For the sum in
  \autoref{prop:degree-is-fibre-counted-with-signs}
  to make sense, we need that $I$ is finite.
  However, this actually follows from $M$ being compact
  by assuming contrarily that $I$ is infinite.

  Without loss of generality, $U$ is open.
  As $N$ is Hausdorff and compact, it is normal,
  and we may separate the closed sets $\set{ x } $
  and $N \setminus U$ by some open neighborhoods,
  say $N \setminus U \subset V$.
  Then, $f^{-1}(V)$ and each of the $U_i$
  form an open cover of $X$.
  But as each of the $y_i$ is contained in only one of these
  by assumption, this cover has no finite subcover, \contra.
\end{remark}

\begin{refproof}{prop:degree-is-fibre-counted-with-signs}
  Consider the diagram
  \[
    \begin{tikzcd}
      &
      H_d\left(\coprod U_i, \coprod U_i \setminus \set{ y_i } \right)
      \ar[swap]{d}{\cong}
      &
      \directsum\limits_{i \in I} H_d(U_i, U_i \setminus \set{ y_i } )
      \ar[swap]{l}{\cong}
      \ar[swap]{d}{\cong}
      \ar[bend left = 80]{dd}
        {\directsum\limits _{i \in I} \left(\frestriction{f}{U_i}\right)_*}
      \\
      H_d(M)
      \ar{r}
      \ar[swap]{d}{f}
      &
      H_d(M, M\setminus f^{-1}(x))
      \ar{d}
      \ar{r}{\cong}
      &
      \directsum\limits_{i \in I} H_d(M, M\setminus \set{ y_i } )
      \ar[dl, rounded corners, blue, to path =
        {
          ([xshift=1em]\tikztostart.north)
          -- ([yshift=2.5em, xshift=1em]\tikztostart.north)
          -- ([yshift=2.5em, xshift=4.5em]\tikztostart.north)
          |- ([yshift=.5em]\tikztotarget.north east)
          -- (\tikztotarget)
        }
      ]
      \\
      H_d(N)
      \ar{r}{\cong}
      &
      H_d(N,N\setminus \set{ x } )
      &
      H_d(U, U\setminus \set{ x } )
      \ar[swap]{l}{\cong}
    \end{tikzcd}
  .\]
  Note that this commutes since all maps are induced
  by the natural inclusions.
  As the middle row is given as the sum of the isomorphism
  $H_d(M) \to H_d(M, M \setminus \set{ y_i } )$, this sends
  $[M]$ to $\sum_{i \in I} μ_{y_i}$.
  But as the $ε_i$ are defined by the map indicated in blue, we also know
  that in total, $H_d(M) \to H_d(N, N\setminus \set{ x } )$ maps
  \[
    [M] \mapsto \sum_{i \in I}μ_{y_i} \mapsto \sum_{i \in I}ε_i μ_x
  \]
  and thus by commutativity of the diagram, $\deg(f) = \sum_{i \in I}ε_i$.
\end{refproof}

\begin{editor}
  At this point of the lecture, we did a quick recap on the properties
  of the cup ($\cup $) and cap ($\cap $) products.
  As this does not blend in too well in the middle of these notes,
  Refer to
  \autoref{sec:recap-cap-cup-product}
  for the corresponding contents.
\end{editor}


\subsection{Poincaré duality}

We will for now just state Poincaré duality.
Later in the lecture, we will actually prove a stronger result.

\begin{theorem}[Poincaré duality]
  \label{thm:poincare-duality}
  Let $M$ be a closed, oriented, $d$-di\-men\-sion\-al manifold 
  with fundamental class $[M]$ and  $N$ be an  $R$-module.

  Then cupping with $[M]$ gives an isomorphism
   \[
     - \cap [M] \colon   H^n(M,N) \xrightarrow{\cong} H_{n-d}(M;N)
  \]
  for all $n\in \mathbb{Z}$.
\end{theorem}

\begin{corollary}
  We have $H^n(M;N) = H_n(M;N) = 0$ for all $n>d$.
\end{corollary}
\begin{proof}
  Follows directly since the negative (co)homology groups of $M$
  vanish.
\end{proof}

\begin{corollary}
  The cohomology rings of $\mathbb{C}\mathbb{P}^n$ and $\mathbb{R}\mathbb{P}^n$
  are given by
  \[
    H^*(\mathbb{C}\mathbb{P}^n;\mathbb{Z})
    \cong
    \faktor{\mathbb{Z}[X]}{(X^{n+1})}
  \]
  with $\deg X = 2$ and
  \[
    H^*(\mathbb{R}\mathbb{P}^n, \mathbb{Z}/2)
    \cong
    \faktor{\mathbb{Z}/2[Y]}{(Y^{n+1})}
  \]
  with $\deg Y = 1$.
\end{corollary}

\begin{proof}
  We prove this for $\mathbb{C}\mathbb{P}^n$ only,
  the proof for $\mathbb{R}\mathbb{P}^n$ works the same way.
  We proceed by induction, so we assume the claim for $\mathbb{C}\mathbb{P}^{n-1}$.

  It follows that $x^{n-1} \in H^{2n-2}(\mathbb{C}\mathbb{P}^n ;\mathbb{Z})$ 
  is a generator, using naturality of the cup product and the
  isomorphism
  \[
    H^k(\mathbb{C}\mathbb{P}^{n-1};\mathbb{Z})
    \xrightarrow{\cong}
    H^k(\mathbb{C}\mathbb{P}^n;\mathbb{Z})
  \] 
  for all $k<2n$. 
  By Poincaré duality, we deduce that
  \[
    x \cap [\mathbb{C}\mathbb{P}^n] \in H_{2n-2}(\mathbb{C}\mathbb{P}^n;\mathbb{Z})
  \]
  is a generator.
  Then by the UCT, we get that
  \[
    \left< X^n, [\mathbb{C}\mathbb{P}^n] \right>
    =
    \left< X^{n-1}, x \cap [\mathbb{C}\mathbb{P}^n] \right>
    =
    \pm 1
  \]
  and thus $X^n$ is a generator.
\end{proof}

\todo{Something on canonical orientation of $\mathbb{C}\mathbb{P}^n$}.

\begin{definition}
  For $X$ with $H_n(X;\mathbb{Z}/2)$ finitely generated for all $n$
  and  $0$ for almost all $n$.
  Then
   \[
    \chi(X) \coloneqq \sum_{i=0}^{\infty} (-1)^i \dim_{\mathbb{Z}/2} H_i(X;\mathbb{Z}/2)
  \]
  is the Euler characteristic.
\end{definition}


\section{Applications of Poincaré Duality}
