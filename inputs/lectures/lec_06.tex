%! TEX root = ../../master.tex
\lecture[]{Mo 25 Apr 2022}{Untitled}


Let $(X,A,B)$ be a triple of spaces with  $A$,  $B\subset X$ open.
Then, from the long exact sequence diagram

\[
  \begin{tikzcd}
    \ldots
    \ar{r}
    &
    H_n(X, A\cap B)
    \ar{r}
    \ar{d}
    &
    H_n(X,A)
    \ar{r}
    \ar{d}
    &
    H_n(A,A\cap B)
    \ar{r}
    \ar{d}{\cong}
    &
    \ldots
    \\
    \ldots
    \ar{r}
    &
    H_n(X,B)
    \ar{r}
    &
    H_n(X,A\cup B)
    \ar{r}
    &
    H_n(A\cup B,B)
    \ar{r}
    &
    \ldots
  \end{tikzcd}
\]
where the isomorphism holds by $B \setminus A \subset B \subset A\cup B$ and excision,
we get the Mayer Vietoris Sequence from last time.

\begin{remark}
  As
  $H_d(M, M \setminus \set{ x } ,R)
  \cong
  H_d(M, M \setminus \set{ x } ; \mathbb{Z}) \tensor _{\mathbb{Z}} R$
  by UCT,
  a generator on the left side corresponds to
  $e\tensor r$ with $r\in R^{\times}$ and
  $e\in H_d(M, M \setminus \set{ x } ; \mathbb{Z})$
  a generator.

  For an $\mathbb{Z}$-orientation $\set{ \mu_x } $ and $r\in R^{\times}$,
  we get an $R$-orientation  $\set{ \mu_x \tensor r} $.

  Note that that $\set{ \mu_x \tensor r } $ is in fact an orientation,
  as we have the commutative diagram
  \[
  \begin{tikzcd}
    H_d(M, M \setminus U; \mathbb{Z}) \ar[swap]{d}{} \ar{r}{} & H_d(M, M\setminus \set{ y } ; \mathbb{Z}) \tensor R \ar{d}{} \\
    H_d(M , M \setminus U; R) \ar[swap]{r}{} & H_d(M, M\setminus \set{ y } ; R)
  \end{tikzcd}
  \]
\end{remark}




\section{The mapping degree}

\begin{definition}[Mapping degree]
  \label{def:mapping-degree-oriented-closed-connected-manifolds}
  Let $M$ and $N$ be oriented, connected, closed,
  $d$-dimensional manifolds and $f\colon M\to N$.
  The \vocab{mapping degree} $\deg(f)\in \mathbb{Z}$
  is defined by as the unique integer such that
  \[
    f_*[M] = \deg(f)\cdot f_*[N]
  .\] 
\end{definition}

\todo{introduce notation here.}

\begin{lemma}
  Let $M$, $N$ and $P$ be oriented, connected, closed,
  $d$-dimensional manifolds and $f\colon M\to N$.
  Let $\overline{M}$ and $\overline{N}$ denote $M$ and $N$
  with the other orientation, respectively.
  Then, the following hold:
  \begin{enumerate}[h]
    \item $\deg(f\colon M\to N) = \deg(f\colon \overline{M}\to \overline{N}) = -\deg(f\colon \overline{M}\to N)$.
    \item The mapping degree is multiplicative,
      that is,
      \[
        \deg(M\xrightarrow{f} N\xrightarrow{g} P)
        =
        \deg(f)\cdot \deg(g)
      \]
    \item If $\deg(f) \neq 0$, then $f$ is surjective.
  \end{enumerate}
\end{lemma}

\begin{proof}
  For $1)$ it suffices to note that $[\overline{M}] = - [M]$.

  For $2)$, by functoriality, we have
  \[
    \deg(g \circ f) [P]
    =
    (g \circ f)_*[M]
    =
    g_*(f_*[M])
    =
    g_*(\deg(f)[N])
    =
    \deg(f)g_*[N]
    =
    \deg(f)\deg(g)[P]
  \]

  For $3)$, assume that  $y\not\in \im f$.
  We will show that $\deg(f) = 0$.
  Consider
  \[
    \begin{tikzcd}
      H_d(M;\mathbb{Z})
      \ar{r}{f_*}
      &
      H_d(N,\mathbb{Z})
      \ar{r}{\cong}
      &
      H_d(N,N\setminus \set{ y } ;\mathbb{Z})
      \\
      \left[M\right]
      \ar[mapsto]{r}
      &
      \deg(f)\left[N\right]
      \ar[mapsto]{r}
      &
      \deg(f)\cdot \mu_y=0
    \end{tikzcd}
  \] 
\end{proof}

\begin{oral}
  One immediately gets that homotopy equivalences between
  oriented, closed, connected, $d$-dimensional manifolds
  are surjective.
\end{oral}

\begin{proposition}
  Let $M$,  $N$ and  $f$ be as before.
  Let  $x\in N$ with a neighborhood $U$ such that
  $f^{-1}(U) = \coprod_{i \in I}U_i$
  such that each $U_i$ is mapped homeomorphically onto  $U$.
  Then, for each $y_i \in f^{-1}(x) \cap U_i$,
  \[
    \begin{tikzcd}
      H_d(M, M \setminus \set{ y } )
      &
      H_d(U_i, U_i \setminus \set{ y_i } )
      \ar[swap]{l}{\cong}
      \ar{r}{f}[swap]{\cong}
      &
      Hvd(U, U\setminus \set{ x } )
      \ar{r}{\cong}
      &
      H_d(N, N\setminus \set{ x } )
      \\
      \mu_{y_i}
      \ar[mapsto]{rrr}
      &
      &
      &
      \pm \mu_x
      =\noloc ε_i
    \end{tikzcd}
  \]
  Then,
  \[
    \deg(f) = \sum_{i \in I} ε_{i}
  \] 
\end{proposition}

\begin{oral}
  This property is not totally off the charts,
  it will usually hold for at least some points.
  Think coverings, for example.
\end{oral}

\begin{remark}+
  One often phrases the statement as
  \enquote{$deg(f) = \abs{f^{-1}(x)}$ with signs}.
\end{remark}

\begin{remark}
  For the sum to make sense, we need that $I$ is finite.
  But actually, this follows from $M$ being compact.
\end{remark}

\begin{proof}
  Consider the diagram
  \[
    \begin{tikzcd}
      &
      H_d\left(\coprod U_i, \coprod U_i \setminus \set{ y_i } \right)
      \ar{d}{\cong}
      &
      \directsum_{i \in I} H_d(U_i, U_i \setminus \set{ y_i } )
      \ar[swap]{l}{\cong}
      \ar{d}{\cong}
      \ar[bend left = 100]{dd}{\text{sum of isos induced by $f$}}
      \\
      H_d(M)
      \ar{r}
      \ar{d}
      &
      H_d(M, M\setminus f^{-1}(x))
      \ar{d}
      \ar{r}
      &
      \directsum_{i \in I} H_d(M, M\setminus \set{ y_i } )
      \\
      H_d(N)
      \ar{r}{\cong}
      &
      H_d(N,N\setminus \set{ x } )
      &
      H_d(U, U\setminus \set{ x } )
      \ar[swap]{l}{\cong}
    \end{tikzcd}
  .\]
  One checks commutativity.
  Now it suffices to track where $[M]$ maps to.
\end{proof}

\subsection{Recap of $\cup $ and $\cap $}

Let $(X,A,B)$ be an excisive triad.
Then, the cup product is a natural morphism
 \[
  \cup \colon H^k(X,A;R) \tensor _R H^l(X,b;R)
  \xrightarrow{} H^{k+l}(X,A\cup B;R) 
\]
that satisfies
\begin{enumerate}[h]
  \item $x\cup y = (-1)^{\abs{x} \cdot \abs{y} } y\cup x$
  \item $(x\cup y)\cup z = x\cup (y\cup z)$ 
  \item $δ(\varphi \cup \psi )
    =
    (δ \varphi )\cup \psi + (-1)^{\abs{\varphi } } \varphi  \cup (δ\psi )$ on chain level.
\end{enumerate}
\todo{different sign convention?}

The cap product is defined by
\[
  \varphi ^k \cap σ^n = \varphi (\frestriction{σ}{[n-k,\ldots,n} \frestriction{σ}{0,\dotsc, n-k}
\]
such that
\[
  \cap \colon H^k(X,A;R)\tensor_R H_{k+l}(X,A\cup B;R)
  \to 
  H_l(X,B;R)
\]
It is natural in the sense that
\[
  x \cap f_*α = f_*(f^*x \cap  α)
\] 
\todo{Hatcher: right modules}
\begin{enumerate}[h]
  \item We have $x \cap (y \cap α) = (x\cup y) \cap α$
  \item On chain level, we have
    \[
      \partial (\varphi  \cap σ) = \varphi  \cap (\partial σ) + (-1)^{\abs{\varphi } \abs{σ} } (δ\varphi )\cap σ
    \]
  \item Let $x = [\varphi ] \in H^k(X;R)$ and $α = [σ] \in H_k(X;R)$,
    then
    \[
      \left< x,α \right> =\varphi (σ)\in R
    .\]
    This yields the cap product if $R$ is connected. 
    Thus
    \[
      \left< x, y \cap α \right>  = \left< x\cup y, α \right> 
    .\] 
\end{enumerate}

\section{Poincaré duality}

We will for now just state Poincaré duality.
Later in the lecture, we will actually prove a stronger result.

\begin{theorem}[Poincaré duality]
  \label{thm:poincare-duality}
  Let $M$ be a closed, oriented, $d$-dimensional manifold 
  with fundamental class $[M]$ and  $N$ be an  $R$-module.

  Then
   \[
     - \cap [M] \colon   H^n(M,N) \xrightarrow{\cong} H_{n-d}(M;N)
  \]
  is an isomorphism for all $n\in \mathbb{Z}$.
\end{theorem}

\begin{corollary}
  We have $H^n(M;N) = H_n(M;N) = 0$ for all $n>d$.
\end{corollary}

\begin{corollary}
  The cohomology rings of $\mathbb{C}\mathbb{P}^n$ and $\mathbb{R}\mathbb{P}^n$
  are given by
  \[
    H^*(\mathbb{C}\mathbb{P}^n;\mathbb{Z})
    \cong
    \faktor{\mathbb{Z}[X]}{(X^{n+1})}
  \]
  with $\deg X = 2$ and
  \[
    H^*(\mathbb{R}\mathbb{P}^n, \mathbb{Z}/2)
    \cong
    \faktor{\mathbb{Z}/2[Y]}{(Y^{n+1})}
  \]
  with $\deg Y = 1$.
\end{corollary}

\begin{proof}
  We prove this for $\mathbb{C}\mathbb{P}^n$ only,
  the proof for $\mathbb{R}\mathbb{P}^n$ works the same way.
  We proceed by induction, so we assume the claim for $\mathbb{C}\mathbb{P}^{n-1}$.

  It follows that $x^{n-1} \in H^{2n-2}(\mathbb{C}\mathbb{P}^n ;\mathbb{Z})$ 
  is a generator, using naturality of the cup product and the
  isomorphism
  \[
    H^k(\mathbb{C}\mathbb{P}^{n-1};\mathbb{Z})
    \xrightarrow{\cong}
    H^k(\mathbb{C}\mathbb{P}^n;\mathbb{Z})
  \] 
  for all $k<2n$. 
  By Poincaré duality, we deduce that
  \[
    x \cap [\mathbb{C}\mathbb{P}^n] \in H_{2n-2}(\mathbb{C}\mathbb{P}^n;\mathbb{Z})
  \]
  is a generator.
  Then by the UCT, we get that
  \[
    \left< X^n, [\mathbb{C}\mathbb{P}^n] \right>
    =
    \left< X^{n-1}, x \cap [\mathbb{C}\mathbb{P}^n] \right>
    =
    \pm 1
  \]
  and thus $x^n$ is a generator.
\end{proof}

\todo{Something on canonical orientation of $\mathbb{C}\mathbb{P}^n$}.

\begin{definition}
  For $X$ with $H_n(X;\mathbb{Z}/2)$ finitely generated for all $n$
  and  $0$ for almost all $n$.
  Then
   \[
    \chi(X) \coloneqq \sum_{i=0}^{\infty} (-1)^i \dim_{\mathbb{Z}/2} H_i(X;\mathbb{Z}/2)
  \]
  is the Euler characteristic.
\end{definition}
\todo{recap this}




\section{Applications of Poincaré Duality}

First, we will show
