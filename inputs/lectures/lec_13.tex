%! TEX root = ../../master.tex
\lecture[]{Mo 23 Mai 2022}{Untitled}


\begin{theorem}
  \label{thm:middle-dimensional-of-boundary-manifold-is-even}
  Let $\mathbb{F}$ be a field.
  Let $W^{2n+1}$ be an $\mathbb{F}$-orientable, compact
  manifold with $\partial W = M$.
  Then $\dim H^n(M; \mathbb{F})$ is even and
  \[
    \dim ( \ker ( i_*\colon H_n(M; \mathbb{F}) \to H_n (W; \mathbb{F}))
    =
    \dim ( \im (i^* \colon H^n(W;\mathbb{F}) \to H^n(M;\mathbb{F}))
    =
    \frac{1}{2} \dim H^n(M, \mathbb{F})
  \]
  Moreover, the intersection form $s_n$ (with induced orientation on $M$)
  of any two classes in
  $\im i^* \subset H^n(M, F)$ is trivial.
\end{theorem}

\begin{proof}
  Consider the following part of Poincaré duality:
  \[
    \begin{tikzcd}
      H^n(W; \mathbb{F})
      \ar{r}{i^*}
      &
      H^n(M ; \mathbb{F})
      \ar{r}{δ}
      \ar{d}{\blank \cap \left[M\right]}
      &
      H^{n+1}(W, M ; \mathbb{F})
      \ar{d}{\blank \cap \left[W,M\right]}
      \\
      &
      H_n(M ; \mathbb{F})
      -\ar{r}{i_*}
      &
      H_n(W ; \mathbb{F})
    \end{tikzcd}
  \]
  Hence $(\im i^*) \cap [M] = \ker i_*$.
  In particular, $\dim ( \im i^*)) = \dim ( \ker ( i_*))$.
  Thus,
  \begin{IEEEeqnarray*}{rCl}
    \rank ( i^* )
    &=&
    \dim \im (i^*)
    =
    \dim \ker (i_*)
    =
    \dim H_n (M ; \mathbb{F}) - \rank (i_*)
    \\
    &\stackrel{\text{UCT}}{=}&
    \dim H_n(M ; \mathbb{F}) - \rank(i^*)
    .
  \end{IEEEeqnarray*}
  The UCT part here comes from the naturality of the UCT,
  giving us the commutative diagram
  \[
    \begin{tikzcd}
      H^n(W;\mathbb{F})
      \ar[swap]{d}{i^*}
      \ar{r}{\cong}
      &
      \Hom (H^n(W;\mathbb{F}) ; \mathbb{F})
      \ar{d}{i_*}
      \\
      H^n(M ; \mathbb{F})
      \ar[swap]{r}{\cong}
      &
      \Hom (H_n(M ; \mathbb{F}) ; \mathbb{F})
    \end{tikzcd}
    .
  \]
  Thus indeed we have
  \[
    \dim H_n(M ; \mathbb{F}) = 2 \rank (i^*)
    = 2 \dim ( \im i^*)
    = 2 \dim ( \ker i_*)
  .\]
  Now, for any $i^*(α)$, $i^*(β) \in \im i^*$,
  to show that $i^*(α) \cup i^*(β)$ is zero, it suffices
  to compute
  \begin{IEEEeqnarray*}{rCl}
    s_n(i^*(α) , i^*(β))
    &=&
    \left< ι^*(α) \cup i^*(β), [M] \right> 
    \\
    &=&
    \left< i^*(a \cup β), [M] \right> 
    \\
    &=&
    \left< α \cup β, \underbrace{i_*[M]}_{=0} \right> 
    = 0
  \end{IEEEeqnarray*}
  \todo{Complete proof, was not done in the lecture.}
\end{proof}

\begin{remark}
  Replacing $\dim$ by $\rank$, we could have also formulated
  the theorem for principal ideal domains,
  in particular for $\mathbb{Z}$-coefficients.
\end{remark}

\begin{corollary}
  If $M$ is the boundary of a compact manifold,
  then the Euler characteristic $\chi(M)$ is even.
\end{corollary}

\begin{proof}
  If $M$ is odd-dimensional,
  then $\chi(M) = 0$ by an Exercise sheet.
  By the same proof, in the even-dimensional case,
  $\chi(M) \equiv \dim (H^n (M ; \mathbb{Z} / 2) \equiv 0 \mod 2$ by
  \autoref{thm:middle-dimensional-of-boundary-manifold-is-even}.
\end{proof}

\begin{corollary}
  The manifolds $\mathbb{R}\mathbb{P}^{2n}$ and $\mathbb{C}\mathbb{P}^{2n}$
  are not boundaries of compact manifolds.
\end{corollary}

\begin{remark}
  One could also say that the Euler characteristic $\mod 2$ is a
  bordism-invariant.
  Beware that the Euler characteristic itself is not,
  consider e.g.~oriented surfaces.
\end{remark}

\begin{corollary}
  If $M$ is the boundary of an oriented,
  compact manifold, then $σ(M) = 0$
  (with the induced orientation).
\end{corollary}

\begin{proof}
  By definition, we can restrict to the case $\dim M \equiv 0 \mod 4$,
  so let $\dim M = 4k$.
  Then $\im i^* \subset H^{2k}(M ; \mathbb{R})$ is a Lagrangian $U$
  for the intersection form $s_n$, i.e. $s_n$ vanishes on
  $\im i^*$ by
  \autoref{thm:middle-dimensional-of-boundary-manifold-is-even}
  and $\dim ( \im i^*) = \frac{1}{2} \dim H^{2k}(M ; \mathbb{R})$.
  Thus, $σ(M) = 0$.

  Let $W^+$ and $W^-$ be the subspaces on which $s_n$ is
  positive/negative definite.
  Then $U\cap W^+ = U \cap W^- = 0$.
  Thus, $\dim W^+$, $\dim W^- \leq \frac{1}{2} \dim H^n(M ; \mathbb{R})$
  and thus $\dim W^+ = \dim W^- = \frac{1}{2} \dim H^n(M; \mathbb{R})$.
\end{proof}

\begin{corollary}
  \label{cor:signature-is-bordism-invariant}
  If $M$ is oriented bordant to $M'$,
  then $σ(M) = σ(M')$.
\end{corollary}

\begin{proof}
  We note that
  \[
    0 = σ ( \partial W)
    = σ ( M \sqcup - M')
    = σ(M) - σ(M')
  \] 
\end{proof}

\begin{remark}
  We can define the oriented bordism ring $\Omega_*$
  with elements given by bordism classes of manifolds.
  Addition is given by disjoint union and multiplication
  by the products of manifolds.
  The ring is graded commutative, since
  $[M \times N] = (-1)^{\dim M \dim N} [ N \times M]$.

  By \autoref{cor:signature-is-bordism-invariant}
  and Exercise 5.4, the signature is a ring homeomorphism
  $\Omega_* \xrightarrow{σ} \mathbb{Z}$.
\end{remark}




\section{Smooth manifolds}

\begin{definition}
  Let $M$ be a $d$-dimensional manifold.
  A \vocab{chart} is a homeomorphism $h\colon U \to U'$
  from an open $U\subset M$ to $U'\subset \mathbb{R}^d$
  (or $\mathbb{R}^d_+$).
  A collection of charts $\set{ h_α \colon  U_α \to U_{α'} } $
  is an \vocab{atlas} if $\bigcup_{α} U_α = M$.

  For two charts $h_α$ and $h_β$, there is a \vocab{transition map}
  \[
    τ_{α,β} \colon h_β \circ h_α ^{-1}
    \colon
    h_α ( U_α \cap U_β) \to h_β ( U_α \cap U_β)
  \] 
\end{definition}

\begin{definition}
  An atlas on $M$ is called \vocab{smooth} if all
  transition maps are smooth, i.e.~in $C^{\infty}$.

  A smooth manifold is a manifold $M$ together with a
  maximal smooth atlas.
\end{definition}

\begin{remark}+
  Note that a smooth atlas determines a maximal smooth atlas
  in a unique way by adding all compatible charts.
\end{remark}

\begin{example}
  Consider $\mathbb{R}^d$ with $\set{ \id_{\mathbb{R}^d} } $.
\end{example}

\begin{example}
  On $S^d \subset \mathbb{R}^{d+1}$,
  we can define charts as follows:

  For $i\in \set{ 1, \ldots, d+1 }$ let
  $U_i^+ \coloneqq \set{ x\in S^d \suchthat x_i > 0 } $
  and $U_i^- \coloneqq \set{ x\in S^d | x_i < 0 }$.

  Consider the charts
  \begin{equation*}
      h_i^{\pm}: \left| \begin{array}{c c l} 
    U_i^{\pm} & \longrightarrow & \mathbb{R}^d \\
    x & \longmapsto &  (x_1, \ldots, \hat{x}_i, \ldots, x_{d+1})
    \end{array} \right.
  \end{equation*}
  with inverses
  \[
    (x_1,\dotsc, x_d)
    \mapsto
    \left(x_1, \dotsc, x_{i-1}, \pm \sqrt{1 - \sum_{j=1}^d x_j^2}, x_i, \dotsc, x_d\right)
  \]
  where of course the chosen sign depends on the sign of $U_i$.

  The transition maps are thus given by deleting an entry and
  adding $\pm \sqrt{1 - \sum x_j^2}$ somewhere else.

  Thus, we get a smooth structure.
\end{example}

\begin{definition}
  A continuous map $f\colon M\to N$
  between smooth manifolds is called \vocab{smooth}
  if for each point $p\in M$ there are charts
  $h\colon U\to U'$, $h'\colon V\to V'$
  with $p\in U$ and $f(p) \in V$
  such that $h' \circ  f \circ  h^{-1}$ is smooth.

  A \vocab{diffeomorphism} is a smooth map with smooth
  inverse.
\end{definition}

\begin{oral}
  Note that this definition is equivalent to saying
  that for all such charts $h$ and $h'$,
  the composition is smooth, since the transition maps
  are smooth.
\end{oral}

\begin{remark}
  For smooth manifolds $M$ and $N$,
  there are canonical smooth structures on $M \coprod N$
  and $M \times N$.
\end{remark}
