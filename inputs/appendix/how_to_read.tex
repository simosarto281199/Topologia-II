\section{How to read these notes}
\label{sec:how-to-read}

These notes are my (personal) notes.
I do not want to make guarantees on correctness or completeness
of these notes.

If you are just interested in the statements of the lecture,
Daniels notes on eCampus (or proper literature)
are probably your better options of reference.
Also note that although try to upload contents as soon as possible
after lectures, usually it takes me a few days until the contents
of a new lecture are presented properly and have been revised by
myself etc.
If you're interested in the lecture of the day, this is also probably
not the right place to look at.

However, I try to give more details in these notes than the uploaded
notes on eCampus and collect e.g.~some oral remarks that Daniel gives
in the lectures.
Also, from time to time, some additional content is available when I
think that some more treatment would be adequate.

This should not be taken as a lack of the lecture but rather as
a representation of my personal learning process while attending the
lecture, as the details I do (not) give largely reflect my personal
stage of understanding.

\subsection*{Numbering}

To distinguish content that was part of the lecture and things that
were the authors sole responsibility, the numbering proceeds as in
the lecture.
Environments not numbered in the lecture will receive
subnumbers in this document so that I can reference them.

Environments starting with a dagger (\enquote{$\dagger$})
represent content that was part of the lecture but was not
treated as this certain environment.
I mainly use these to number and reference stuff.

Environments starting with a star (\enquote{*})
mark content that was not part of the lecture and thus added
by the author.
Be especially careful concerning potential errors in such sections.

\subsection*{Contribute}

Error corrections or any other suggestions regarding these notes are welcome
at the projects
\href{https://gitlab.com/latexci/lecture-notes-bonn/topology-2}
{GitLab page}.

I also want to thank Daniel for the excellent lecture he gives,
his notes and the support he shows for these notes to be typed.
