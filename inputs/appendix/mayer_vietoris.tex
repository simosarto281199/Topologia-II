\section{Mayer Vietoris}

This section has been added to give a short overview of Mayer Vietoris.
Recall the abstract nonsense version of Mayer Vietoris in a purely
category-theoretic context:

\begin{lemma}*
  \label{lm:mayer-vietoris-abstract-nonsense}
  Let $C_\chainbullet $ and $D_\chainbullet $ be long exact sequences,
  and left $f_\chainbullet \colon C_\chainbullet  \to  D_\chainbullet $
  be a chain map such that $f_{3n}$ is an isomorphism for all $n$.

  Then, there is a long exact sequence, the Mayer-Vietoris sequence,
  given by
  \[
  \begin{tikzcd}[column sep = 5em]
    C_{3n+2}
    \ar{r}{\partial _{3n+2} \oplus f_{3n+2}} &  
    C_{3n+1} \oplus D_{3n+2}
    \ar[draw=none]{d}[name=X,anchor=center]{}
    \ar{r}{\partial _{3n+2} - f_{3n+1}} &  
    D_{3n+1}
    \ar[rounded corners,
              to path={ -- ([xshift=2ex]\tikztostart.east)
                        |- (X.center) \tikztonodes
                        -| ([xshift=-2ex]\tikztotarget.west)
                        -- (\tikztotarget)}]{dll}[at end]{\partial } \\
    C_{3n-1}
    \ar{r}{\partial _{3n-1} \oplus f_{3n-1}} &
    C_{3n-2} \oplus D_{3n-1}
    \ar{r}{\partial 3n-1 - f_{3n-2}} &
    D_{3n-2}
  \end{tikzcd}
\]
  where $\partial$ is given as the composition
  \[
  \begin{tikzcd}
    D_{3n+1} \ar{r}{\partial _{3n+1}} &  D_{3n} & C_{3n} \ar{l}{\cong}[swap]{f_{3n}} \ar{r}{\partial _{3n}} &  C_{3n-1}
  \end{tikzcd}
  \]
\end{lemma}

\begin{proof}
  See last semesters note, e.g.~available at
  \href{https://latexci.gitlab.io/lecture-notes-bonn/TopologyI/2021_Topology_I.pdf}{GitLab},
  or the hand-written notes by Daniel linked on eCampus.
\end{proof}

We want to give the following version that has not been covered last term:

\begin{theorem}[Mayer Vietoris]
  \label{thm:mayer-vietoris-union-intersection-of-subspaces}
  Let $(X,A,B)$ be an excisive triad in the sense of
  \autoref{def:excisive-triad}.
  Then, there is long exact sequence in homology given by
  \[
    \begin{tikzcd}[column sep = tiny]
    &
    \ldots
    \connectingmark[Y]
    \ar{r}
    &
    H_{n+1} ( X, A\cup B)
    \connectingmap[Y]{\partial}
    \\
    H_n ( X, A\cap B)
    \ar{r}
    &
    H_n ( X, B; R) \oplus H_n(X, A)
    \connectingmark[Z]
    \ar{r}
    &
    H_n ( X, A\cup B; R )
    \connectingmap[Z]{\partial}
    \\
    H_{n-1} ( X, A\cap B)
    \ar{r}
    &
    \ldots
    ,
    \end{tikzcd}
  \]
  where the boundary morphism is given as the composition
  \[
    \begin{tikzcd}[column sep = tiny]
      H_n(X,A\cup B)
      \ar{r}
      &
      H_n(A\cup B, B)
      &
      \ar[swap]{l}{\cong}
      H_n(A, A\cap B)
      \ar{r}{\partial }
      &
      H_{n-1}(X,A\cap B)
    \end{tikzcd}
  \]
  with the boundary operator of the triple sequence.
\end{theorem}

\begin{proof}
  We use the triple sequences of the spaces
  $(X,A\cap B, A)$ and $(X, A\cup B, B)$,
  giving us a map of long exact sequences
  \[
    \begin{tikzcd}
      \ldots
      \ar{r}
      &
      H_n(X, A\cap B)
      \ar{r}
      \ar{d}
      &
      H_n(X,A)
      \ar{r}
      \ar{d}
      &
      H_n(A,A\cap B)
      \ar{r}{\partial }
      \ar{d}{\cong}
      &
      \ldots
      \\
      \ldots
      \ar{r}
      &
      H_n(X,B)
      \ar{r}
      &
      H_n(X,A\cup B)
      \ar{r}{\partial }
      &
      H_n(A\cup B,B)
      \ar{r}
      &
      \ldots
    \end{tikzcd}
  \]
  This commutes as all maps are induced by the
  natural inclusions.
  The isomorphism is just given by the excision assumption.
  Thus, by
  \autoref{lm:mayer-vietoris-abstract-nonsense},
  the claim follows.
\end{proof}
